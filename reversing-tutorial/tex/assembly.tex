\section{Looking at the assembly of a binary}

\textbf{However, this is only scratching the surface of information we can extract from this binary. \newline
From here you start learning about the two core ways you can go about reverse engineering a program (Static
Analysis/Dynamic Analysis) and two of the most powerful tools we recommend you use for both (IDA Pro: Static, Radare2:
Dynamic). These 2 approaches and their associated tools are at the absolute core of Rev/Exploitation, which is why we
will be hitting these concepts in depth.\newline
\newline
\subsection{Static Analysis vs Dynamic Analysis}
In the simplest explanation I can think of, Static Analysis will show you all the information about your program that
can be seen without running it and Dynamic Analysis is everything you can learn about the code by running it and seeing
how it behaves. For me it helps to understand this using an example with high level code like C. \newline
\newline
If someone gave you the C source code to a program and told you to explain to them what it does or you will be fired,
how would you do that? Logically, I can think of two ways I would go about doing that. One way would be to open up the
c file, read it step by step and try to get a picture of what it does as a whole. The second way would be to compile
and run it, see what happens, how it behaves, and maybe connect some dots to get the gist of the program. Now already
we can tell that each one has its pros and cons. }

\textbf{For option one, the pro would be that we get to see everything the program might do; all the variables,
functions, and therefore, possible outcomes. You just have to put the work in and take the time to read every line of
that code. But what if the program is gigantic or if there are changes that happen to the program that will only happen
once you run it, like one that starts by decrypting itself. This is where option 2 would be necessary. }

{\centering
\textbf{\newline
For option two, running the program, we will not as easily be able to view all the inner workings of our program, but it
often shows us enough information to get a strong understanding in a much quicker time. \newline
}}

\subsection{Intro to Static Analysis}

\textbf{Let's open it in IDA Pro. Open IDA(Type into your terminal }cd \~{} \textbf{then }idademo70/ida )\textbf{ , then
select our binary when prompted. \newline
Once IDA finishes loading, it should look like this: }

{\centering   
\sitgfx[width=5.3957in,height=5.052in]{reversing-img009.png}{fig:unk}{TODO CAPTION}
 \par}
\textbf{What does all this crap mean?? At least we can see our string ``}Hello World!\textbf{{}''. \newline
We're going to take a break from the idea of hello world and dig into some
}\textbf{\textcolor[rgb]{0.07450981,0.30980393,0.36078432}{assembly.}}\textbf{ The words in the left column are the
names of }\textbf{\textcolor[rgb]{0.21960784,0.4627451,0.11372549}{instructions.}}\textbf{ For now, ignore everything
except the following block:}

{\centering
\textbf{\newline
\newline
 }  
\sitgfx[width=4.8126in,height=1.0209in]{reversing-img010.png}{fig:unk}{TODO CAPTION}
 
\par}

\textbf{Let's start at the ``call'' instruction and work backwards. You might be able to guess what the functionality of
this is after we google for what ``puts'' does.\newline
\newline
}\textit{Function: int puts (const char *s)}

\textit{The puts function writes the string s to the stream stdout followed by a newline. The terminating null character
of the string is not written. (Note that fputs does not write a newline as this function does.)}

\textit{puts is the most convenient function for printing simple messages. \newline
For example:puts ({\textquotedbl}This is a message.{\textquotedbl});}

\textit{outputs the text `This is a message.' followed by a newline.\newline
\newline
}\textbf{So, we can be pretty confident that ``}call puts'' \textbf{...calls puts. But puts takes an argument! Puts
takes a pointer to a char (this is how strings are denoted in C
}\url{https://en.wikipedia.org/wiki/Null-terminated_string}\textbf{ .) If we recall our programming 2 pointers, we know
that pointers point to memory locations(}\textbf{\textcolor[rgb]{0.21960784,0.4627451,0.11372549}{addresses}}\textbf{)
where the actual value can be found. \newline
The instruction on the same line as our string, }{}``lea''\textbf{ has the format lea operand1, operand2. ``LEA'' stands
for ``Load Effective }\textbf{\textcolor[rgb]{0.21960784,0.4627451,0.11372549}{Address}}\textbf{{}'' and puts the
}\textbf{\textcolor[rgb]{0.21960784,0.4627451,0.11372549}{address}}\textbf{ of operand2 into operand1. (Nearly every
x86 instruction has the format }instruction operand1, operand2. \textbf{Usually it's referred to as }destination,
source\textbf{ instead of }operand1, operand2.)\textbf{  \newline
So with all of this in mind, we can be fairly confident that a pointer to the string ``Hello World!'' is getting stored
in operand1, which is }\textbf{\textcolor[rgb]{0.21960784,0.4627451,0.11372549}{edx}}\textbf{.
}\textbf{\textcolor[rgb]{0.21960784,0.4627451,0.11372549}{Edx }}\textbf{is a
}\textbf{\textcolor[rgb]{0.21960784,0.4627451,0.11372549}{register}}\textbf{. The CPU has a lot of registers, and
registers are small bits of fast storage located on the physical CPU. If you remember from earlier, this program is a
32-bit executable which runs on a CPU with 32-bit registers! Nearly every CPU instruction has a register involved.}

{\centering
  
\sitgfx[width=1.9583in,height=1.8752in]{reversing-img011.png}{fig:unk}{TODO CAPTION}
 \par}
\textbf{So we know that a pointer to (the address of) our string gets put into edx and that puts gets called. And when
we run the program, we know that ``hello world!'' gets printed. That encapsulates the functionality of our hello world
program. }

\subsection{Analyzing logmein lvl1(aka log)}

\textbf{Now that we're a little familiar with the basics, let's look at something a little
}\textbf{\textit{harder}}\textbf{. This program came from the wild (}\textbf{\textcolor{red}{or a CTF maybe)}}\textbf{
and we don't know anything about it besides the name - log. What can we do to figure it out?From your terminal, cd to
Desktop/rev\_be101/binaries by issuing the command }cd Desktop/rev\_be101/binaries\textbf{.
}\textbf{\textcolor[rgb]{0.21960784,0.4627451,0.11372549}{File }}\textbf{says its a 32-bit executable for linux, and
}\textbf{\textcolor[rgb]{0.21960784,0.4627451,0.11372549}{strings }}\textbf{has some interesting output. Do these on
your own. \newline
\newline
Now let's open it in IDA. Type into your terminal }cd \~{} \textbf{then }idademo70/ida -\textbf{you can use the Tab key
to autocomplete. Once IDA gets open, agree to the license agreement and then choose to open a new file. Navigate to
Desktop/rev\_be101/binaries and open our file }log. \textbf{We'll land in }\textbf{\textit{main }}\textbf{ {}- if for
some reason you navigate away from this screen, double click ``main'' on the functions screen. If your graph view turns
funky, try pressing the spacebar to get back to the graph.}  
% Unhandled or unsupported graphics:
\sitgfx[width=6.5in,height=3.2638in]{reversing-img012.png}{fig:unk}{TODO CAPTION}
 

\textbf{Let's scroll down a bit with the arrow keys or mouse drag (about 2 boxes, to the circled portion) until we see
this:  }

  
\sitgfx[width=6.5in,height=3.4862in]{reversing-img013.png}{fig:unk}{TODO CAPTION}
 

\textbf{From this one screenshot we }\textbf{\textit{can}}\textbf{ see pretty much the entire functionality of this
program, without even having had to run it!!! That is, if we look closely and can read assembly. Let's direct our focus
to the ``call'' instructions. If you click once on the word ``call'' IDA will highlight all of the appearances for you.
Take a look at your own IDA and think about what this all represents. It's okay if you're mostly lost! Once again,
keeping up with the theme of }\textbf{\textit{reverse}}\textbf{, we're going to be working backwards. \newline
\newline
We can see a few ``call'' instructions. The first code block should look somewhat familiar from the hello\_world
example, except instead of edx and puts, it's eax and printf. Go back and look at that again if you don't remember what
that looked like.}

  
\sitgfx[width=6.5in,height=4.611in]{reversing-img014.png}{fig:unk}{TODO CAPTION}
 \textbf{\newline
The second call is to scanf, another familiar function from C. We see that \%s gets put into eax and pushed before scanf
gets called. Before that, a }\textbf{\textcolor[rgb]{0.21960784,0.4627451,0.11372549}{local variable
}}\textbf{(var\_30) gets put into eax and pushed. Notice how ``push eax'' happens right before(sometimes multiple
times) each call in the above screenshot? The address of a string gets put into eax and then pushed before printf, then
the addresses of a local variable and another string (the format string ``\%s'') gets put into eax and then pushed, and
finally the addresses of two }\textbf{\textcolor[rgb]{0.21960784,0.4627451,0.11372549}{local variables }}\textbf{get
put into eax before getting pushed before check\_password gets called. If you guessed that these values that get pushed
are the arguments for these functions, you'd be right. And we know all without ever running the program! Let's go ahead
and run the program now with our newfound knowledge and see what happens. (ctrl+c exits the currently running
program)\newline
}  
\sitgfx[width=6.5in,height=1.139in]{reversing-img015.png}{fig:unk}{TODO CAPTION}
 

\textbf{Do you see where printf and scanf come into play? And can you see where check\_password happens? Please go back
to IDA. At the end of the code block with the ``incorrect. Try again:'' message, there is an arrow that goes back up to
the block containing check\_password(). This is indicative of a loop, as we saw when we ran the program. The program
prompts for input, checks it, and if it's wrong, says ``Incorrect. Try Again:'' and waits for more input.  }
\sitgfx[width=6.5in,height=3.028in]{reversing-img016.png}{fig:unk}{TODO CAPTION}
 \textbf{\newline
Feel free to keep running the binary and trying to guess the password. We're going to find it ourselves. But first, some
more assembly. \newline
}

\clearpage{\centering
\textbf{Let's take a look at the last two instructions of the block containing check\_password. There's }\textit{test
eax,eax}\textbf{\textit{ }}\textbf{and} \textit{jz short loc\_5C1}\textbf{. The }\textbf{\textit{test
}}\textbf{instruction (for our purposes) checks if operand1 + operand2=0 by performing a bitwise AND on operand1 and
operand2. If they are, it sets the Zero Flag to 1. Otherwise, it sets the Zero Flag to 0. The Zero Flag is part of the
EFLAGS register, specifically the ZF bit. So the next instruction, jz(jump if Zero Flag is set to 1), checks the Zero
Flag, and if it's set to 1, it takes the green path. If it's set to 0, it takes the red path. Because the last function
called before these instructions was check\_password, we can assume that check\_password stores its result in eax. So
if check\_password returns 1, we take the red path(win). If it returns 0, we take the green path(lose/try again). Even
though we haven't even figured out how to win, we know so much about the program! \newline
}  
\sitgfx[width=6.4583in,height=4.5728in]{reversing-img017.png}{fig:unk}{TODO CAPTION}
 \textbf{\newline
Now let's get into the meat of the program. Double-click on the word check\_password on the graph, or double click on it
from the functions window (it's right below main). Take a minute or two to familiarize yourself with the structure of
the function, and recall everything we know about what happens with this function. We just figured out that if
check\_password returns 0, we lose and try again, and if it returns 1, we win! Let's start at the very end of the
function, the win and lose conditions, and work our way back. }  
% Unhandled or unsupported graphics:
\sitgfx[width=6.5in,height=3.8055in]{reversing-img018.png}{fig:unk}{TODO CAPTION}
 
\par}

\textbf{Let's look at the code block circled. We know that the function check\_password returns its value in eax, since
the }\textbf{\textit{test}}\textbf{ instruction from before was }\textbf{\textit{test}}\textbf{ing eax,eax. We get more
proof of that now, where eax is the last register to be modified before the retn (return) instruction. It appears we've
found the }\textbf{\textit{lose/try again}}\textbf{ condition! We can see that the top block circled ``mov''{}'s(the
}\textbf{\textcolor[rgb]{0.21960784,0.4627451,0.11372549}{mov}}\textbf{ instruction copies operand2 into operand1) 0
into [ebp+var\_8] which is the location of a local variable. The next block of code }mov's\textbf{ that same
[ebp+var\_8] into eax, before returning. So, we can safely assume that [ebp+var\_8] is the location of the return
value.\newline
}(In Intel x86 assembly notation, the brackets [] denote the memory contents at a specific address. This is important
because \textit{ebp} stands for \textit{base pointer} - it points to the \textit{base address} of The
\textcolor[rgb]{0.21960784,0.4627451,0.11372549}{Stack}. So [ebp +var\_8] is the compiler's way of telling the CPU
``[go to this memory address and get whatever is there.]'')

\textbf{So now we know how to lose. Can we find how to win? If we continue working backwards from the return point, we
see that red arrow that also points to the return block. Double click it or follow it back up.}

{\centering   
\sitgfx[width=3.1146in,height=4.4272in]{reversing-img019.png}{fig:unk}{TODO CAPTION}
 \par}
\textbf{The light blue arrow is the same one previously mentioned. Could it be the win condition?? Let's take a closer
look at the code block above it.}

{\centering   
\sitgfx[width=4.6354in,height=1.5in]{reversing-img020.png}{fig:unk}{TODO CAPTION}
\par}
\textbf{We see some old friends, and some new ones. This }\textbf{\textit{test}}\textbf{ instruction looks kind of
familiar but also kind of wrong! }  
\sitgfx[width=6.5in,height=4.8752in]{reversing-img021.png}{fig:unk}{TODO CAPTION}
 

\textbf{It appears AL is the lowest 8 bits of the EAX register. So }\textit{test} al, al\textbf{ is really
}\textit{test} the lowest 8 bits of eax, the lowest 8 bits of eax\textbf{! But what could be in eax at this point in
the program? Do you have any guesses? Recall that when we ran the program on our command line, it asked for input and
told us that we were wrong. We know how to be wrong, but how can we be right? This function compares our input string
to another that is located somewhere in the program. We could continue statically analyzing the function and we would
find the location of the victory string, but we would never be able to determine exactly what the password is. For
this, we turn to dynamic analysis. We'll be using a tool called radare2, a combined disassembler and debugger. } 

\sitgfx[width=6.5in,height=4.3335in]{reversing-img022.png}{fig:unk}{TODO CAPTION}
 \textbf{To open this file in debug mode in radare2, type `}r2 -d log'\textbf{, then after it gets open, `}aaa'\textbf{
and then `}afl\textbf{{}'. Aaa tells r2 to analyze all functions, and afl tells it to list the analyzed functions.}

  
\sitgfx[width=6.5in,height=4.1807in]{reversing-img023.png}{fig:unk}{TODO CAPTION}
 

  
\sitgfx[width=6.5in,height=4.9028in]{reversing-img024.png}{fig:unk}{TODO CAPTION}
 \textbf{\newline
We see some old and tired friends! Type `}s main' \textbf{to instruct r2 to seek to main, then `}db main' \textbf{and
`}dc\textbf{{}' to instruct radare to set a breakpoint at main and then continue until the breakpoint.}

\textbf{Now type ``V'' then enter to open visual mode}

  
\sitgfx[width=6.5in,height=4.2362in]{reversing-img025.png}{fig:unk}{TODO CAPTION}
 

\textbf{The first visual mode is a basic hexdump of the binary. This can be useful at times but we will go for a
different view for now. type ``p'' twice without hitting enter. Anything look familiar? It's the same disassembly from
IDA, just now in radare's format with some bonus info.}

  
\sitgfx[width=6.5in,height=4.7362in]{reversing-img026.png}{fig:unk}{TODO CAPTION}
 

\textbf{This may be a little bit to take in if this is completely new to you so take a deep breath, but don't worry.
It's not so bad. Now, you can see a view of the instructions in memory that are going to be executed at the bottom of
the display. The rows of 0x00000000's with labels are the
}\textbf{\textcolor[rgb]{0.21960784,0.4627451,0.11372549}{registers }}\textbf{of the computer and at the top of the
screen is a view of the }\textbf{\textcolor[rgb]{0.21960784,0.4627451,0.11372549}{stack}}\textbf{.}

  
\sitgfx[width=5.8335in,height=3.6457in]{reversing-img027.png}{fig:unk}{TODO CAPTION}
\textcolor{red}{***** Don't know much about how a Stack works? Read below. If you are well informed, skip
this*****}\newline
\textbf{The Stack}

\textbf{The stack is a programming structure in memory that works as a first in, last out queue. Items as ``pushed'' on
the stack in a certain order and then ``popped'' off in the reverse order. So, for example, say my stack starts out
with a 1.}

\textbf{0x0010: 1}

\textbf{and then I push a two(2) and then a three(3) onto the stack \newline
in that order - push 2 \newline
\ \   {}- push 3}

\textbf{0x0008: 3\newline
0x0009: 2\newline
0x0010: 1}

\textbf{Now If I were to pop something off the stack, I would first get a three (since it is on top of the stack) and
the stack would then look like}

\textbf{0x0009: 2\newline
0x0010: 1}

\textbf{Inside of your computer, the stack works the same way. The numbers pushed and popped off the stack of a 64 bit
computer can be variables, memory addresses, and other data and will look more like 0x00007f11e8a9bf2a (a memory
address) or 0x656b6f4456964f (A piece of inputted user data) This process is referred to as changing the stack frame.}

\textbf{One caveat of the stack is that in computer memory it grows upward toward }\textbf{\textit{lower memory
addresses}}\textbf{ so when you push numbers onto the stack the value of the topmost data on the stack is at the lowest
memory address:}

\textbf{0x0008: Topmost value on stack is at lowest address (0x0008)\newline
0x0009:\newline
0x0010: lowest value of stack is at the highest address (0x0010)}

\textcolor{red}{******************END OPTIONAL STACK EXPLANATION HERE****************************}

\textbf{For those new to assembly, registers are like variables for assembly. You can move data into and out of
registers (mov eax, 6)and use them to push and pop data to and from the stack (pop rax). Several registers have special
purposes. }\textbf{\textcolor[rgb]{0.21960784,0.4627451,0.11372549}{eip}}\textbf{ is used to point to the next assembly
instruction the computer is going to execute in memory. Right now, assuming you typed }s main\textbf{, }db
main\textbf{,and }dc\textbf{ above, }\textbf{\textcolor[rgb]{0.21960784,0.4627451,0.11372549}{eip}}\textbf{
is0x565d054d.}\textbf{\textcolor{red}{ }}\textbf{(The IP of EIP stands for Instruction Pointer.)}

\textbf{\textcolor{red}{ }}\textbf{If we look at the code at the bottom panel we see this agrees with the topmost
instruction in memory.}

  
\sitgfx[width=6.5in,height=5.6665in]{reversing-img028.png}{fig:unk}{TODO CAPTION}
 

\textbf{The register esp is used to point to the top of the stack. As you can see, the address in esp is 0xffd3008c
which agrees with the highest stack address in the top view.}

  
\sitgfx[width=6.5in,height=5.1252in]{reversing-img029.png}{fig:unk}{TODO CAPTION}
 

\textbf{main is the execution starting point of programs written in C and C++ for command line. the ``db'' command sets
a breakpoint inside the code of the program. So by typing ``db main'' you set a breakpoint at the beginning of the main
function, which is the start of the program. (Radare2 actually starts a program and breaks before the main function or
the program starts.) Other functions can be used the same way. For example if a program had a function in it called
sayHello() then ``db sayHello'' would put a breakpoint at the beginning of that function. You can enter offsets into
the ``db'' command as well. You can type ``db main+1'' to set a breakpoint at the second instruction of the main
function. The ``dc'' command stands for debug continue, and runs the program until another breakpoint is hit.
Breakpoints can also be set by using the up and down keys(in visual mode) to look through memory and then hitting shift
+ b to place a breakpoint at the highest instruction in the display. (You cannot use this when the tab for command
execution is open at the bottom of the screen. hitting enter without typing any command will get rid of the tab.)
Hitting shift + b again will remove the breakpoint. or using a minus (-) in the ``db'' command. For example ``db
-main'' will remove the breakpoint at the start of main.}

\textbf{Press the `}:\textbf{{}' character while in Visual Mode to bring up the command entry. Let's set a breakpoint on
the call to check\_password by bringing up command entry with :, then typing }db sym.check\_password \textbf{(you can
use tab to autocomplete here as well).  If you run ``dc'' now, you will see the execution of the program. First,
``Please enter a word or character:'' is printed to the screen}  
% Unhandled or unsupported graphics:
\sitgfx[width=6.5in,height=5.3335in]{reversing-img030.png}{fig:unk}{TODO CAPTION}
 

\textbf{I gave it the word ``asdf''. I pressed enter after typing asdf just like when we ran the program before, and I
get the message hit breakpoint at: 565d0625. Press enter to return to visual mode. Did you notice the change? Instead
of being in }main\textbf{ we're now in }sym.check\_password\textbf{! }  
% Unhandled or unsupported graphics:
\sitgfx[width=6.5in,height=5.2083in]{reversing-img031.png}{fig:unk}{TODO CAPTION}
 

\textbf{There are some other things to note here as well. In the stack view(top block) , at the very end of the visible
stack, we see our input string! Asdf! ...and there's another suspicious string in the stack view too. Let's set a
breakpoint after the program initializes its local variables (}:, \textbf{then }db 0x5664a63c, \textbf{then press
enter- pro tip- select the memory address with your mouse and middle click it to copy and paste in one click!). Then,
press }: \textbf{and type }dc \textbf{then press enter to continue to the next breakpoint. R2 should tell you ``hit
breakpoint at }0x5664a63c.\textbf{{}'' }  
\sitgfx[width=6.5in,height=5in]{reversing-img032.png}{fig:unk}{TODO CAPTION}
 

\textbf{Let's get debugging. Press ``s'' to single step through the program. You will notice the highlighted memory
address will move down to the jmp instruction. Press ``s'' again and we will end up in a block of familiar code!} 

\sitgfx[width=6.5in,height=3.778in]{reversing-img033.png}{fig:unk}{TODO CAPTION}
 

\textbf{Let's take a step away from the flow of the program to examine the values in the registers. Before we execute
any of this block, we see that there are some values currently stored in eax and edx, but  they don't really have a lot
of meaning to us at this moment. Press ``s'' twice so that the highlighted instruction is: \newline
}0x565cf672  \ \ 01d0  \ \ add eax, edx\newline
  
\sitgfx[width=6.5in,height=1.4445in]{reversing-img034.png}{fig:unk}{TODO CAPTION}
 

\textbf{We see that both edx and eax have new values! Edx got 0x000000000 (aka 0) and eax got 0xffb57da8! What the heck
is 0xffb56da8?? It looks kind of like the esp..but remember in hexadecimal we count 8,9,a,b,c,d,e,f. And the base
pointer (the bottom of the stack) is 0xffb57d}\textbf{\textcolor{red}{68}}\textbf{...so this memory address is below
the stack for this function. \newline
We can get the data at this memory address by pressing `:' and issuing the command ``}pxw 40 @ eax'' \textbf{or ``}pxw
40 @0xffb57da8''\textbf{. They're the same thing, since eax = 0xffb57da8. This is similar to how the brackets [] work
in x86 assembly. When we type }pxw \_nbytes\_ @ memory \textbf{, it's just like saying ``get nbytes from [memory].''We
see our input string that we entered when the program was in main!!!!  Let's keep going.}\newline
   
\sitgfx[width=6.5in,height=1.389in]{reversing-img035.png}{fig:unk}{TODO CAPTION}
 

\textbf{Press ``s'' 2 more times so that the highlighted instruction is  \ \ }

0x565cf677\ \ 84c0  \ \ test al, al

  
\sitgfx[width=6.5in,height=2.5138in]{reversing-img036.png}{fig:unk}{TODO CAPTION}
\textbf{The first time we pressed s just now, nothing changed. It added edx to eax, which was adding 0 to a number, no
effect. The next instruction is more interesting. Wise persons might recognize, even without knowing what the movzx
instruction does, that eax contains the hex representation of an ASCII value. Some quick googling reveals that eax now
contains the ASCII representation of the letter ``a'', Hx 61 in the chart. We even reverse engineered the functionality
of }  
\sitgfx[width=6.3543in,height=0.2291in]{reversing-img037.png}{fig:unk}{TODO CAPTION}
 !\newline
\textbf{We figured out that it moves 1 byte (the size of a char in C) from operand2 into operand1.  More specifically,
in this instance it moves a byte from the address that eax currently holds ([}0xffb57da8] = asdf = 0x66647361\textbf{)
and puts it into eax. Let's keep going. Press ``s'' through the test and jump instructions(2x) ,and then 3 more times
so we're looking at this screen.}  
\sitgfx[width=6.5in,height=1.8335in]{reversing-img038.png}{fig:unk}{TODO CAPTION}
 

\begin{center}
 
  \sitgfx[width=1.6043in,height=4.9272in]{reversing-img039.png}{fig:unk}{TODO CAPTION}

\end{center}
\textbf{We see the same functionality from the last block we analyzed, albeit with some more instructions. This time, it
movzx's a byte (char) from our input string into edx, instead of eax. Step through this instruction and verify for
yourself.\newline
Cool. Let's keep stepping to try to figure out what happens next. It moves a local variable, the same ones from before,
4h, which we know is 0, into ecx. Verify that as well. The next instruction is juicy. One of the function arguments
gets moved into eax! Let's do} : \textbf{then }pxw 50 @ eax.   
% Unhandled or unsupported graphics:
\sitgfx[width=6.5in,height=4.5138in]{reversing-img040.png}{fig:unk}{TODO CAPTION}
 

\begin{center}
 
  \sitgfx[width=1.8543in,height=0.2189in]{reversing-img041.png}{fig:unk}{TODO CAPTION}

\end{center}
\textbf{EAX now holds the memory address of (a pointer to) a string that we did not enter!! So EDX holds the address of
our input string and EAX holds a foreign string. What happens next?After a useless }add eax,ecx\textbf{, our new friend
movzx moves a byte(character) from the string pointed to by eax into eax! }  
% Unhandled or unsupported graphics:
\sitgfx[width=6.5in,height=2.0555in]{reversing-img042.png}{fig:unk}{TODO CAPTION}
 

\textbf{So now we have our 0x61 (lowercase A) in edx, and 0x55 (capital U) in eax. Can you guess what the next
instruction, }cmp dl, al \textbf{does? If your answer was }  
% Unhandled or unsupported graphics:
\sitgfx[width=6.5in,height=1.8055in]{reversing-img043.png}{fig:unk}{TODO CAPTION}
 

\textbf{You'd be correct! The cmp instruction sets the Zero Flag to 1 only if the two operands are equal. So, if the
character of the input string, `a' and the character of the password string, U, are the same, then we get to take the
jump. Feel free to keep stepping through the program and repeating our commands, but I think we've done enough
reversing on this binary. Let's run it again, this time giving it what we think the password is. In r2, press q to
return to the r2 console and then type qy to quit radare2. Run log (./log) and give it the password!! }
\sitgfx[width=6.4165in,height=1.3126in]{reversing-img044.png}{fig:unk}{TODO CAPTION}
\url{https://www.youtube.com/watch?v=5pZa8YobAeg}
