\section{Why do we Reverse?}
\textbf{Reverse Engineering in infoSec, as simply as we can put it, is to take a
}\textbf{\textcolor[rgb]{0.21960784,0.4627451,0.11372549}{binary, }}\textbf{which is just }

\textbf{another word for a computer program, and figure out what it's trying to do. You run binaries all the time in day
to day life - when you run your browser, when you run Microsoft Word, when you check the weather on your phone. Every
time you run a program or application, a binary file made of 1s and 0s tells your device exactly what it needs to do.
\newline
\newline
If you have written code before, you know that before it can do anything, you have to turn that into a program/binary so
that you can use it. Before that your code is just a useless file. But what if you want to know what's happening under
the covers when you use Word, or Chrome, or anything else running on your computer? This is done only by Reverse
Engineering.  \newline
\newline
Let's start up our Kali Linux and write up our own code get a better idea. The Linux terminal allow you to access most
of the applications in your environment from the command line. For windows users, it is equivalent to the command
prompt application (cmd). Access your terminal by searching ``Terminator'' in the Activities section of the toolbar
}or\textbf{ selecting the red icon on the dash that looks like }
\includegraphics[height=0.2689in]{reversing-img001.png}
 \textbf{. Here you should see a window open with your current user and location in the file system.\newline
}

\subsection{Programming and Compiling Hello World}

{\centering
\sitgfx[width=6.5in,height=1.778in]{reversing-img002.png}{fig:unk}{TODO CAPTION}
 \par}
\textbf{Here you can see my terminal open on the machine name(kali:\~{} root\#), kali, as user root, with \~{} symbol
letting me know I am in the home directory. Pressing Ctrl+L will clear the screen for me whenever I need.\newline
}

\textbf{Let's start by writing a simple ``hello world'' program in the C programming language. You can use any text
editor you like, but I'll use }\textbf{\textcolor[rgb]{0.21960784,0.4627451,0.11372549}{Leafpad }}\textbf{cause it's
simple.\newline
Type in terminal:}\textit{ leafpad hello.c\newline
\ \ }  
\sitgfx[width=4.8693in,height=3.1602in]{reversing-img003.png}{fig:unk}{TODO CAPTION}
 \textbf{ }

\textbf{This opens a text editor for me to write my program in. So let's do that. Write it yourself, even if you are not
good at C I suggest you learn enough to do this so you understand what each line does. Your code should look like
this:}

\textbf{\newline
\ \ }  
\sitgfx[width=4.3854in,height=1.302in]{reversing-img004.png}{fig:unk}{TODO CAPTION}
 

\textbf{Now let's turn our code into a binary(which will be 32-bit, not 64, for reasons later explained).}

  
\sitgfx[width=6.5in,height=0.5555in]{reversing-img005.png}{fig:unk}{TODO CAPTION}
 \textbf{\newline
Listing (ls) the content in our file location now shows two hello files: \newline
{}-hello.c: our C source code\newline
{}-hello\_binary: our binary created from our source file\newline
\ \ }  
\sitgfx[width=4.7602in,height=0.6146in]{reversing-img006.png}{fig:unk}{TODO CAPTION}
 

\textbf{Running (./) hello\_binary achieved our objective, outputting the string we printed}

\textbf{What happens when the }\textbf{\textcolor[rgb]{0.21960784,0.4627451,0.11372549}{compiler}}\textbf{, a program
that interprets source code and outputs machine code, interprets our source code? Basically, A's and B's go through a
process and become 1's and 0's that make it possible for your processor (CPU) to understand. In fact, it's a little
less abstract than simply 1's and 0's. More on that later. }

\textbf{The compiler goes through your source, line by line, and tries to figure out what you meant. It turns your
source code into }\textbf{\textcolor[rgb]{0.21960784,0.4627451,0.11372549}{instructions}}\textbf{, or specific tasks
for your CPU to execute. You can't tell your CPU ``x = 5'' so your compiler of choice will tell it a series of
}\textbf{\textcolor[rgb]{0.21960784,0.4627451,0.11372549}{instructions }}\textbf{that represent ``x=5''.}


\subsection{We don't always get source code with our programs.}

\textbf{\newline
So what would happen if I deleted the source code for hello.c and renamed the binary? Would you have a clue what this
program even does until running it? And even then, your understanding of this program will be based on just the
behavior you can see when doing so. It is this fact that allows products like Chrome and Word to be shared to you
without making it easy for you to figure out EXACTLY how it works and sell your new product based off of this
information.  But there is still a way see what's inside for those who have the skill. Enter Reverse
Engineering...\newline
\newline
{}-How would we see how it works? The }\textbf{\textcolor[rgb]{0.21960784,0.4627451,0.11372549}{file }}\textbf{command
gives us some basic information about the file hello\_binary. ELF = Executable and Linkable Format. An .exe on Windows
is functionally equivalent to an ELF on Linux. The other thing to note on the output of
}\textbf{\textcolor[rgb]{0.21960784,0.4627451,0.11372549}{file}}\textbf{ is the phrase ``Intel 80386'' which is the CPU
}\textbf{\textcolor[rgb]{0.21960784,0.4627451,0.11372549}{architecture, }}\textbf{aka the physical hardware that this
program was compiled to run on. More specifically, 32-bit, which refers to the size of the
}\textbf{\textcolor[rgb]{0.21960784,0.4627451,0.11372549}{registers }}\textbf{of said CPU(more on those in a bit).}

\textbf{ }  
\sitgfx[width=6.5in,height=0.6252in]{reversing-img007.png}{fig:unk}{TODO CAPTION}
 

\textbf{So that's cool, but what else is in this program? The
}\textbf{\textcolor[rgb]{0.21960784,0.4627451,0.11372549}{strings }}\textbf{command lists all of the human-readable
strings in the binary of length 4 or greater. \newline
}  
\sitgfx[width=6.5in,height=3.5138in]{reversing-img008.png}{fig:unk}{TODO CAPTION}
 \textbf{\newline
We'll note that there are a lot of strings for such a tiny program! About 3 quarters of the way down, note our string
``hello world!'' from the source code. The compiler and version we used are a little bit below it. A lot of these
results are actually just to are left from the compilation process. We used the function ``printf'' which on linux is a
member of}\textbf{\textcolor[rgb]{0.21960784,0.4627451,0.11372549}{ glibc}}\textbf{ (the standard library for C
programs on Linux). So, the compiler had to bring in glibc in order to know what we meant by printf, hence all the
glibc/libc related strings.
}
