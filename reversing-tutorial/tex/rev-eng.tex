\section{Reverse Engineering -- Static Analysis Walkthrough}

\textbf{Static analysis is done to help you understand how the program should function before running it. The purpose of
static analysis is to use the program flow and the assembly code to understand how the program should run and how to
find out which sections are important and should be further analyzed. This is important because some programs use many
jumps and blocks of code, and it is important to see how the program flows before jumping into debugging. On top of
this, static analysis does not require a run-time environment, so it can be done from any machine. I use my faster Mac
host instead of my boxes, but any operating system will work!}

\textbf{To start, we will be statically analyzing a binary using a tool called IDA. IDA is seen as an industry standard
for static analysis because it has tons of useful plugins, many different features, and even the ability to script with
Python. However, this is why the professional version costs a few hundred thousand dollars. We will only be using some
of the more simple and useful features for this tutorial, and all of these can be found in the free IDA demo that is
provided.}

\textbf{The program that we are analyzing is called
}\textbf{\textcolor[rgb]{0.21960784,0.4627451,0.11372549}{example}}\textbf{, and we plan on discovering what this
program does. It was written in C and compiled with GCC with the m32 flag, for compatibility on 32 bit machines. Rather
than showing you the C code that was used, we will use the program flow to determine the C code from the assembly
language.}

\textbf{To open a file in IDA, we will launch IDA. You will see a screen asking whether you would like to disassemble a
new file, work on your own, or load a previous disassembly. This is your first time disassembling this file, so you
will click the box that says ``New''. It will then present a box of options that affirms the format of the 32 bit ELF
file. Make sure that IDA will auto-comment, and select ``OK'' to proceed. After some loading as IDA evaluates the
program, you should get a screen that looks like the following:}

  
\sitgfx[width=6.5in,height=4.0626in]{reversing-img045.png}{fig:unk}{TODO CAPTION}
 \textbf{This is the graph mode of IDA, and it presents the functions as series of connected boxes presented as jumps.
The other main view of IDA is known as the text view, and it presents all of the information as it is loaded into RAM.
This is more confusing to read and involves the format of the ELF file and all of it's sections, so we will start by
using Graph view. If you are not already in this view, press the spacebar to switch between these two main views.}

\textbf{In this view, there are a few key features to look at for this program. On the left, there is an area for the
functions, known as the ``functions window''. In this, all of the functions that are used in the binary can be seen.
However, the majority of them are functions used by GCC and by the stdio.h file that is imported. This can be
determined by the functions with the purple shade, like \_strcmp, \_gets, \_puts, \_strlen, and libc\_start\_main.}

\textbf{It is important to determine which functions are used because they can indicate how the program works. In this
case, we can guess that two strings are compared due to the strcmp method, and that we will give input with gets and
see output with puts. The strlen could be used to find the length of a string that we enter or a string that is
hardcoded in memory. All of the other functions that are used take a part in making sure that the ELF can run
independently, and they help link all of the parts of the file and libraries together in virtual memory. This is a
complicated process, but luckily these functions don't affect the flow of the program that we can affect.}

\textbf{On top of this, it is always helpful to look at the strings that can be found in the program before we analyze
further. IDA can look for areas in memory in which a series of consecutive memory falls into the printable range of
ASCII. In other words, it looks through the whole file for strings of words and numbers that may end up being printed
on the screen or used in various functions. To open this view, click on ``Views'' in the toolbar, then ``Open
subviews'' and ``Strings''. The keyboard shortcut for the strings subview is Shift-F12. Once you open the strings
subview, you should see the following screen:}

  
\sitgfx[width=6.1252in,height=3.8335in]{reversing-img046.png}{fig:unk}{TODO CAPTION}
 \textbf{Most of these strings are function names, and they are linked to the libc functions that we import in the
stdio.h file that I previously mentioned. These all have an address that begins with LOAD, as can be seen. This means
that they are used when the program loads libc and imports the functions and references that will be used. After these,
there are 5 strings that stand out. The last string is stored in the eh\_frame, as can be seen in its address, and it
comes from the exception handling part of the ELF file, so it is not relevant in the execution that we can see. By now,
you can probably see that ELF files and other executables are way more complicated than they seem in the C code that is
used to create them, and IDA shows all of the sections used by the compiler to make sure that this code can run in the
computer and load into its RAM properly. It's a very intricate and entertaining process, but it is usually not too
important when reverse engineering. I would not recommend affecting any of this data because it could destroy the
program, for all practical purposes.}

\textbf{Now let's look at the 5 strings that seem interesting. The first says, ``If you managed to get here, you used an
exploit!'' That sounds like a place in the program that is not meant to be reached and can only be reached with an
exploit, but we are not focused on exploitation yet. I will reference that later in this tutorial, after understanding
the basic flow and how to reverse engineer the program in the way that it is supposed to function. The second string
says, ``Please enter the password:''. This seems like a prompt that would be printed before we give input. The third
strings says, ``itisnotthiseasy''. That one seems confusing, but it may be useful. The fourth string says, ``Correct!''
We can assume that this is the part of the program that we would like to reach, because you may assume that this
relates to the password that it asks us to enter. The fifth string seems like the other possibility, as it states,
``Mission failed, we'll get them next time''. You may have noticed that all of these have addresses that begin with
``.rodata'', which stands for ``read only data.'' This means that these strings can only be read by the program, and
they are located in an area of memory that cannot be written to or executed. Now, it is time to check out the flow!}

\textbf{Click on ``IDA View-A'' in the list of tabs to go back to the graph view. Now we will read through the graph and
attempt to understand what the program should do. Starting from the top box, a few definitions are created for you by
IDA. The local variables and labelled ``var\_10'' and ``var\_C''. These variables are created and used inside of this
function. They are stored in lower memory addresses than EBP, so IDA can recognize that a repeated negative offset from
EBP indicates a local variable in memory. The arguments are labelled ``argc'', ``argv'', and ``envp'', and they have
positive offsets from EBP. You may recognize these because they are commonly referred to in C programs where the main
function has command line arguments. However, you can look through the graph to find out where any of these arguments
are referenced, and it can be seen that they are only declared and never used. On top of this, ``s'' is defined, and
IDA commonly uses the offset ``s'' to refer to strings. In the first block everything up to ``lea'' is used to set up
the stack and position the memory, so this can be better analyzed in dynamic analysis. }  
% Unhandled or unsupported graphics:
\sitgfx[width=6.4583in,height=2.4791in]{reversing-img047.png}{fig:unk}{TODO CAPTION}
 \textbf{Once we reach the lea, knowledge of x86 assembly language becomes incredibly important. I always try to keep a
reference manual at hand in case I do not recognize an instruction, but Google often turns out to be the biggest help.
Lea stands for ``load effective address'', so it pretty much functions as a way to store a pointer. In this case, the
pointer for some offset of register ebx is stored in register eax, and luckily IDA simplified life by displaying the
string that is at that section of data. The pointer to this string is stored in register eax, and the next instruction
``push''es this pointer onto the stack. After this, the ``puts'' call is performed, in which the puts function of the
stdio.h library is called. The libc manual says that ``puts'' takes one argument, a string pointer, and it prints this
memory to standard out, or the screen. IDA recognized that eax was on top of the stack and commented that the value in
eax is the argument passed to the puts call. Any blue phrase that begins with a semicolon in IDA is an indication of an
argument for the next mentioned function call. We have already determined that eax held the pointer the string ``Please
enter the password:'', so this string must be printed!}

\textbf{ \ \ After this, 10h (10 in hex) is added to esp to move the stack to clear the frame created for the puts call,
and then 0Ch is subtracted to make room for the next call. Don't worry about not understanding how the stack works too
much in this tutorial because it is much easier to visualize in dynamic analysis, and it is not worth stressing about
or getting caught up on. However, it is helpful to change the 10h and 0Ch into decimal to be more easily read. To do
this, select the green text for either number and type ``H''. This will convert from hexadecimal to decimal format, and
it is now much easier to read. Now, let's look at the next function call. The address of ebp+s is loaded into eax, and
then eax is pushed. After this, gets is called. We know that gets uses one argument, which is the pointer to the area
in memory in which the string is to be stored. Now we know that location ``s'' is where our string that we input is
stored. This must be out ``password''!}

\textbf{ \ \ For the next function call, the address of our password is loaded again into eax and pushed before strlen
is called. This must determine the length of the string that we input. After the stack pointer is added again, a mov
instruction is used. This instruction moves the value stored in one register or area of memory into another. IDA uses
Intel syntax, so you read it from right to left. The value of eax is moved into ebp+var\_10, and eax is the standard
register for return value. This means that var\_10 is our string length! Let's rename it! To do this in IDA, click on
the green ``var\_10'' and type n. We will name it input\_length. Once you do this, you will realize that all occurences
of ``var\_10'' have been replaced with ``input\_length''. 0 is moved into var\_C, so we do not know what it's function
is yet. After this, jmp is used to reference an unconditional jump, and we leave the box. Boxes in IDA end with jumps
or endpoints.}

\textbf{ \ \ In the next box, var\_C is moved into eax, and this is compared with input length, and if it is less, the
green jump is taken. Decisions are being made, which seems scary! But don't worry, there are only two options for the
flow at this time. The first time, unless no input is given, 0 must be less than the size. This means we will follow
the green arrow to the next box. }  
\sitgfx[width=2.5937in,height=2.3335in]{reversing-img048.png}{fig:unk}{TODO CAPTION}
 

\textbf{We will approach this section like we did with the previous boxes. At first, the pointer of the string is loaded
into edx, and var\_C is moved into eax. These are added together and stored in eax (remember, we read from right to
left for Intel and IDA uses Intel syntax). Now let's look at all that we did so far and try to represent it in C code:
var\_C = var\_C + s*. The next instruction looks ugly, but we can tell that it involves moving. The byte referenced to
by the pointer at eax is moved into eax, and an Intel manual shows that the upper 3 bytes of register eax are zeroed.
Then, 1 is added to eax, and the contents of register eax are moved into ecx. This is confusion, but we will go back to
this and why it happens. Then, the address of string s is loaded into edx and var\_C is moved into eax again, and they
are added together and stored into eax again. After this, register cl (the least significant byte of register ecx) is
moved into the memory referenced to by eax, and then var\_C is incremented.}

\textbf{Hmm, if var\_C is incremented each time, starting at 0 and ending at the length of the string, this seems like a
for loop. Indeed, in the C code this was a for loop! We can rename var\_C ``counter'' to simplify our reading. To
understand what the box we just read does, it helps to know that a character is one byte in size. From the first half
and the C code that we created, we can tell that the pointer to each character consecutively in the loop is loaded into
eax, and then a byte at that location is moved to eax. This must be the characters, from the first character we enter
to the last! Then, 1 is added to this value, so the character at that value is incremented. After this, the code
repeats and our C code line that we created is used again to reference the location of the character. However, the
character in ecx is moved back into memory. This means that the characters are each incremented, starting with the
first and ending with the last! It may seem tricky at first, but the ability to recognize loops and recognize the
traversal of arrays becomes more natural with time. If you are confused you should read again after knowing the
outcome, it may make more sense.}

\textbf{Now that we have traversed the word, we can follow the red arrow.}

  
\sitgfx[width=6.4898in,height=3.948in]{reversing-img049.png}{fig:unk}{TODO CAPTION}
 \textbf{Firstly, the pointer to the string ``itsnotthiseasy'' is loaded into eax, which is then pushed onto the stack.
This is referenced to as argument s2. The string in memory that we entered and that had each character incremented by 1
is then loaded into eax and pushed onto the stack as argument s1, and strcmp is called. This compares the two strings,
returning 0 if they are equal and other numbers if they are not. Test eax, eax tests to see if the value in eax is zero
or not. If it is zero, it sets a special zero flag. If this zero flag is not set, the green arrow is taken, and the
string ``Mission failed, we'll get them next time.'' is printed. This is a long string, so you can double click
``aMissionFailedW'' to go to that section of memory and read the whole string.}  
% Unhandled or unsupported graphics:
\sitgfx[width=6.5in,height=0.6563in]{reversing-img050.png}{fig:unk}{TODO CAPTION}
 \textbf{27h is used to refer to the single quote because that is it's ASCII value and it is a special character in this
case. This allows it to be used in the string even if it is used to denote the start or stop of a string here. To go
back to the previous part, click on the back facing arrow above the function list.}

  
\sitgfx[width=4.1252in,height=2.1772in]{reversing-img051.png}{fig:unk}{TODO CAPTION}
 \textbf{\newline
This example of stepping backwards in our analysis can also be seen as a usage of a cross reference. A cross reference
is a location in the program in which this function or string is called and used, and it can be a useful way of finding
when functions and strings are used in the flow of the program. To find the cross references to this string, you can
right click on it and select ``List cross references to.'' This can also be done with the keyboard shortcut ``ctrl-x''.
When you do this, you should see the following:}

  
\sitgfx[width=6.4898in,height=1.6043in]{reversing-img052.png}{fig:unk}{TODO CAPTION}
 

\textbf{This is the location in the graph in which we found this string, because it was the code that loaded the pointer
to the string into register eax! Double click this to go back to the graph view of our function, and we can continue
from there.}

\textbf{ \ \ The final block helps end program execution by loading 0 into eax and popping all of the variables off of
the stack. The 0 in eax is a standard in programming to represent no errors in execution, and it is used because in C
programs, the main method is forced to return an integer as it is declared ``int main()''. We have statically analyzed
the whole program!}

\textbf{ \ \ Now, let's look at the bigger picture. We give a password to the program, every character is incremented by
1, and then it is compared to the string ``itsnotthiseasy.'' To do this successfully, we should decrement all of the
characters of ``itsnotthiseasy'' to get the correct password! This password would be
``hsrmnssghrd{\textasciigrave}rx'', as can be seen on an ASCII chart. We cannot test during static analysis, but we can
try to understand what should happen and what we would expect to happen. This makes dynamic analysis easier because we
know what to try and which parts to check.}

\subsection{Dynamic Analysis}

\textbf{We will use radare2 as a tool to dynamically analyze the program now. While IDA has some debugging features,
radare2 (commonly referred to as r2) has tons of features to view the variables and stack as you debug a binary, and it
is incredibly helpful in understanding what happens as a binary runs.}

\textbf{One easy way to get to a point in a binary is by manipulating the jumps. Sometimes, it does not matter what you
enter because you can change the register values later on to reach your goals. We will load the binary example into
radare2 and seek to main.}

  
\sitgfx[width=6.5in,height=4.0626in]{reversing-img053.png}{fig:unk}{TODO CAPTION}
 \textbf{Now, recall the IDA graph that we analyzed, and remember that the final jump to either correct or failure.
Before this jump is taken, the ``test eax, eax'' instruction is used to determine if the value in eax is zero or not,
setting the zero flag. Use the arrows to scroll through the main function in radare2 and set a breakpoint at the ``test
eax, eax'' instruction. Now, continue and hit this point.}

  
\sitgfx[width=6.5in,height=4.0626in]{reversing-img054.png}{fig:unk}{TODO CAPTION}
 \textbf{We will change the value in the register eax to be zero. This way, the JNE instruction (jump if not equal to)
will evaluate to false and not be taken. This should print the affirmative response. To do this, type ``:dr eax=0''.
You can see the value of the eax register change to 0. }  
\sitgfx[width=6.5in,height=4.0626in]{reversing-img055.png}{fig:unk}{TODO CAPTION}
 

\textbf{After you do this, step twice to see if the jump is taken or not. You will see that the jump is not taken! By
changing the value in the register you have beaten the challenge! Continue execution to see that you won.}

\textbf{Wait, is it also possible to change the instructions? Yup! This time we will change the JNE instruction into a
NOP, and this will make sure that the jump is not taken and the program flows into the correct section of the program.
To do this, you will load example into radare2 and seek to main. Then, you will scroll through main until the ``jne
0x56624689'' instruction is at the top of the code. Then, you can use ``c'' to go to cursor mode, and tab twice to put
the cursor in the code. Use the arrow keys to put the cursor on the opcodes of the instruction that you want to edit.
Now, type ``a'' and you will be prompted to enter assembler opcodes.  You will write two NOP instructions (0x90) in the
place of the jump not equal instruction.}

  
\sitgfx[width=6.5in,height=4.0626in]{reversing-img056.png}{fig:unk}{TODO CAPTION}
 

  
\sitgfx[width=6.5in,height=4.0626in]{reversing-img057.png}{fig:unk}{TODO CAPTION}
 

\textbf{We use two because a NOP is one byte and a JNE is 2 bytes in length. A NOP instruction stands for ``no
operation'', and the code execution flows through these NOP instructions without changing anything except for the eip.
This is called patching, and it is used to change the flow of programs and redirect execution to wherever you want.
Doing this in radare2 in debug mode will not change this permanently, however, so you can test anything that you want
without having to worry about ruining your binary! As you run the program, you can put any input and still reach the
correct component of the program.}

\textbf{Now, we will put this binary into a different context. Let's say that this is a challenge and the flag is the
password that we input. Now, the input actually matters, so we must find out what the password should be. Open the
binary example in radare2 and seek to main. Here we can see the gets, puts, strlen, and strcmp functions that were seen
in IDA as they are called. They are highlighted into a light green color at the default color setting to stand out as
functions that are called.}

  
\sitgfx[width=6.5in,height=4.0626in]{reversing-img053.png}{fig:unk}{TODO CAPTION}
 

\textbf{In order to see what the puts function is doing, let's put a breakpoint after puts and debug continue until that
point. After doing this, you will see the phrase ``Please enter your password: ``, which shows the string that the puts
call used from the stack as an argument}

\textbf{The string was located at the address [ebx-0x189f]. The square brackets dereference the value inside them in x86
assembly. So if ebx was 0x55555555 then [ebx] would tell what is stored at the address 0x55555555 instead of the
location itself. To display the string at other times during execution before ebx is changed, you can print the string
in the memory at this location using the command ``pxw 0x20 @ ebx-0x189f''. Note that the pxw dereferences the argument
given to them automatically. So to see the value at ebx use ``pxw 0x20 @ ebx'', not ``pxw 0x20 @ [ebx]''. Pxw stands
for print hexadecimal (x) word. A word is 4 bytes in length. }

  
\sitgfx[width=6.5in,height=4.0626in]{reversing-img058.png}{fig:unk}{TODO CAPTION}
 

\textbf{After this, we know that the gets function takes user input because it is a commonly used libc function. To find
out what the purpose of a libc function like gets is, you should use the command ``man gets'' in the command line in a
new terminal window. You should notice that the man page warns the user that gets is an insecure function to use.
Looking back before the gets function is called, we see that the value ebp-0x24 is loaded into eax and pushed to the
stack. This means that the local variable ebp-0x24 is the first argument to gets, and the location of the local
variable where our input is stored. This can be determined by the man page for gets as the first argument, so it is
always a good idea to use the man pages for unknown libc functions. Single step to the call.}

  
\sitgfx[width=6.5in,height=4.0626in]{reversing-img059.png}{fig:unk}{TODO CAPTION}
 

  
\sitgfx[width=6.5in,height=4.0626in]{reversing-img060.png}{fig:unk}{TODO CAPTION}
 

\textbf{Right now, you should use shift s to step over the gets call. This is important because the code for the gets
function is long and complicated. If there is an issue and you land inside of a function like gets and just want to
return to the previous function, use the command dcr to continue until the next ret instruction. Instead of this, you
can use s main to seek back to main, and you can set a breakpoint after gets returns. Then, dc should return you to
this point.}

\textbf{Now let's print 20 bytes, as words, at the location ebp -- 0x24 to check that our input is stored there. After
this, let's try to find the precise location that the input is stored at. To do this, you can use the command ``? ebp
-- 0x24''. The second value shown is the hexadecimal formatted address for this value. As you can tell, this is the
same value as eax, which shows that the gets function returns the value of the pointer to the input.}

  
\sitgfx[width=6.5in,height=4.0626in]{reversing-img061.png}{fig:unk}{TODO CAPTION}
 

\textbf{Note that the same ebp-0x24 value is the argument to sym.imp.strlen. place a breakpoint after the call to strlen
and debug continue. Note that after the call, the value in eax is 0x10 (which is 16, not 10). The this value is place
into the stack variable ebp-0x10. On the left side of the screen, you'll see two arrows. These show that the program
execution will be jumping up and down the instructions here. If you look at the bottom jump command, you notice that
the compare jumps if the value in ebp-0xc is lower then the value in ebp-0x10. So this loop will probably count upwards
by one until it reaches the length of our string.}

  
\sitgfx[width=6.5in,height=4.0626in]{reversing-img062.png}{fig:unk}{TODO CAPTION}
 

\textbf{Looping at the code inside the loop, we see that the loop takes the address of our string and counts upward
through each letter in the string, and adds one to the ascii value in each letter (so `h' becomes `I', `a' becomes `b',
`b' becomes `c').}

  
\sitgfx[width=6.5in,height=4.0626in]{reversing-img063.png}{fig:unk}{TODO CAPTION}
 

\textbf{Next we see that the binary is calling a string compare. Let's see what is being compared to our input.}

  
\sitgfx[width=6.5in,height=4.0626in]{reversing-img064.png}{fig:unk}{TODO CAPTION}
 

\textbf{Well, now we think we may know what the password entered should be. Now, I'll reset the instruction pointer to
the beginning of main by executing dr eip = 0x565fd5dc. (This address may be different for you on your computer.)}

\textbf{To solve the crackme, we will use a python script. We know that the solution should involve every character of
itsnotthiseasy that is decremented by one. The python script is as follows:}

  
\sitgfx[width=6.5in,height=4.0626in]{reversing-img065.png}{fig:unk}{TODO CAPTION}
 

\subsection{Tips and Tricks}

\textbf{ \ \ Now that you have gone through a normal static analysis workflow, you can tell that this could become
incredibly time consuming very quickly. However, there are tons of little tricks that can save you time. Many
constructs are used in similar fashions throughout many C programs, like the way that all for loops are recognizable
for their arrows backwards to previous boxes and a register or variable incrementing. To help you see this, I created a
C program that involved many different data types and control flows. This is similar to a great video I watched by a
Youtuber known as Live Overflow, so I want to make sure to give credit to him and his ideas. I highly recommend that
you watch his binary exploitation series. Because this is not as much about reversing a program without source code, I
provided the source code  for the program. It is in the c\_constructs.c file, which you should have open next to IDA as
you analyze.}

  
\sitgfx[width=3in,height=4.1252in]{reversing-img066.png}{fig:unk}{TODO CAPTION}
 

\textbf{To help you understand what the size is of each character, you can see what kind of memory location they are
stored in. A dword, or double word, is a memory location that is 4 bytes in length. A qword, or quad word, is a memory
location that is 8 bytes in length. As you can tell from the various pointers to memory location, all of which were
created for local variables, all are stored in dwords, qwords, or bytes. The most common of these that you will see are
characters in strings and integers. The character is always represented as a byte, and integers and stored as dwords in
memory. Because of this, if you see bytes being used and referenced frequently in a program, you could infer that
strings and characters are probably being used. If there are dwords being used in a password checking program, it is
more likely that it is some sort of passcode with numbers.}

  
\sitgfx[width=6.4898in,height=3.802in]{reversing-img067.png}{fig:unk}{TODO CAPTION}
 

\textbf{This is an example of a for loop. The variable labelled ``var\_40'' is the counter that loops through the for
loop. When you look quickly at a loop like this, you can tell that this is a loop because of the blue arrow that points
upwards to the box at loc\_563. Also, you can tell that it's a for loop that increments from 0 to 9 because the initial
instruction is ``mov [ebp+var\_40], 0''. Then, it is compared against 9 after being moved into edx, and it is
incremented in the box on the bottom left of the picture above. All of these can be seen by reading through the
assembly language line by line, but looking at the boxes before and after the arrow that goes upwards can reveal a ton
about the functionality.}

  
\sitgfx[width=6.5in,height=3.9374in]{reversing-img068.png}{fig:unk}{TODO CAPTION}
 

\textbf{Here are two other loop examples. The loop on the right is taken first because var\_40 is still 10 from the
previous for loop. Here a while loop is used, which can somewhat be seen by the way that 2 was added to edx instead of
1, but for loops can use increments of 2 as well. You can definitely tell that it is a loop because of the arrow
pointing upwards, and it isn't quite the same as the for loop from before. Then, the box on the left is a do while
loop. This can be seen because of the arrow pointing upwards into the same box. Also, the instructions that start with 
{}``f'' indicate floating point instructions, and these are used when floats or doubles are used in computations.}

  
\sitgfx[width=6.4898in,height=0.6252in]{reversing-img069.png}{fig:unk}{TODO CAPTION}
 

\textbf{A switch can be easily distinguished by the long row of boxes. All of these boxes are aligned and are referenced
from the same box above them. Also, I can identify switches often based on the fact that they are reached by a ``jmp
eax'' instruction rather than a jump based on the comparison of values. Many different values can be stored in eax, and
these pointers can be used based on the case of the switch.}

  
\sitgfx[width=2.8646in,height=1.2083in]{reversing-img070.png}{fig:unk}{TODO CAPTION}
 

\textbf{When a function is called in this program, the add function, the arguments are pushed onto the stack in reverse
order. This is a normal calling convention for C, and it is called the cdecl convention. This stands for ``C
declaration'', and it is used in most C programs that are compiled on a Unix system as ELF files. All of the programs
that we have reversed so far have used the cdecl convention, like the main methods in this program and in the previous
example program. What happens in these is that the arguments are pushed in reverse order after space is created for
these, and then they are located in higher memory than the stack pointer esp once the function is called and the new
stack frame is created. }

  
\sitgfx[width=3.3646in,height=3.7602in]{reversing-img071.png}{fig:unk}{TODO CAPTION}
 

\textbf{In the called function, the new frame is created without subtracting esp at all. This is because no new
variables were created in this function, and the return value was put into eax. This is a standard convention in
assembly language, so you can safely assume for most conventions that eax holds the return value for the previously
called function until anything else is placed in the register.}

  
\sitgfx[width=6.5in,height=4.5311in]{reversing-img072.png}{fig:unk}{TODO CAPTION}
 

\textbf{Here you can see how the hardcoded array is placed into memory. All of the integers are placed into memory
locations located next to each other, within 4 bytes of each other. This helps in traversing the array because the
elements are all 4 bytes away from each other. An important line here is ``mov [ebp+eax*4+var\_58], edx''. This helps
locate each element in the array because the first element of the array is located at ebp+var\_58. Then, the eax times
4 helps traverse from one element to the next based on the value of eax, starting with eax equal to 0. After this, a
string is hardcoded into the array based on the ascii values in reverse order, known as little endian order. Also, as
we mentioned the return values for functions, a value of 0 is moved into eax to indicate that the program was able to
exit without any error!}

\textbf{The best way to learn reverse engineering, especially static analysis, is by creating your own programs with
strange and unique constructs and seeing how the graph in IDA reflects your functions. Knowledge of loops and
conditions makes this much easier, but there are many ways in which different algorithms and functions could be
represented. To practice your reverse engineering skills and learn how program control flows in various ways, I have
provided a famous bomb lab from Carnegie Mellon University. This would be tested by running it or using dynamic
analysis, but you can use static analysis to read each phase and understand how they work. The most important thing is
to be patient, keep an x86 manual nearby, and make sure to look at the bigger picture!}
