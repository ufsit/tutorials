% This file was converted to LaTeX by Writer2LaTeX ver. 1.6
% see http://writer2latex.sourceforge.net for more info
\documentclass[letterpaper]{article}
\usepackage[ascii]{inputenc}
\usepackage{amsmath}
\usepackage{amssymb,amsfonts,textcomp}
\usepackage[T1]{fontenc}
% For meeting senior design specifications
\usepackage{parskip}
\usepackage[english]{babel}
\usepackage{color}
\usepackage[top=0in,bottom=1in,left=1in,right=1in,includehead,head=1in,headsep=0.9602in]{geometry}
\usepackage{array}
\usepackage{hhline}
\usepackage{hyperref}
\usepackage{float}
\hypersetup{colorlinks=true, linkcolor=blue, citecolor=blue, filecolor=blue, urlcolor=blue}
\usepackage{graphicx}

\title{}
\author{}
\date{}

% Keep our graphics nice and neat
\graphicspath{{img/}}

% Custom figure macro for quick and easy changes across the document
%   \ucffig
%     Param 1: label
%     Param 2: figure content
%     Param 3: caption
\newcommand{\sitfig}[3]{
\begin{figure}[H]
\centering
\makebox[\textwidth][c]{
#2
}
\label{#1}
\end{figure}
}

\newcommand{\sitgfx}[4][scale=1.0]{
\sitfig{#3}{\includegraphics[#1]{#2}}{#4}
}

%%%%%%%%%%%%%%%%%%%%%%%%%%%%%%%%%%%%%%%%%%%%%%%%%%%%%%%%%%%%%%%%%%%%%%%%%%%%%%%%%%%%%

\begin{document}

% TITLE
\pagecolor{black}
\begin{center}
  \Huge
\color{white}
\textbf{Reverse Engineering \& Exploitation Trek-through} \\[1cm]

\huge
Owen Flannagan, Ambrose Douglas, \\ Nick Lang \& Elan Rasabi \\[5.7in]

\normalsize
Version 1.1 - \today \\
\href{http://ufsit.org}{\tt \color{white}http://ufsit.org}
\end{center}

\newpage
\pagecolor{white}

\tableofcontents

\section{Why do we Reverse?}
\section{Why do we Reverse?}
\textbf{Reverse Engineering in infoSec, as simply as we can put it, is to take a
}\textbf{\textcolor[rgb]{0.21960784,0.4627451,0.11372549}{binary, }}\textbf{which is just }

\textbf{another word for a computer program, and figure out what it's trying to do. You run binaries all the time in day
to day life - when you run your browser, when you run Microsoft Word, when you check the weather on your phone. Every
time you run a program or application, a binary file made of 1s and 0s tells your device exactly what it needs to do.
\newline
\newline
If you have written code before, you know that before it can do anything, you have to turn that into a program/binary so
that you can use it. Before that your code is just a useless file. But what if you want to know what's happening under
the covers when you use Word, or Chrome, or anything else running on your computer? This is done only by Reverse
Engineering.  \newline
\newline
Let's start up our Kali Linux and write up our own code get a better idea. The Linux terminal allow you to access most
of the applications in your environment from the command line. For windows users, it is equivalent to the command
prompt application (cmd). Access your terminal by searching ``Terminator'' in the Activities section of the toolbar
}or\textbf{ selecting the red icon on the dash that looks like }
\includegraphics[height=0.2689in]{reversing-img001.png}
 \textbf{. Here you should see a window open with your current user and location in the file system.\newline
}

\subsection{Programming and Compiling Hello World}

{\centering
\sitgfx[width=6.5in,height=1.778in]{reversing-img002.png}{fig:unk}{TODO CAPTION}
 \par}
\textbf{Here you can see my terminal open on the machine name(kali:\~{} root\#), kali, as user root, with \~{} symbol
letting me know I am in the home directory. Pressing Ctrl+L will clear the screen for me whenever I need.\newline
}

\textbf{Let's start by writing a simple ``hello world'' program in the C programming language. You can use any text
editor you like, but I'll use }\textbf{\textcolor[rgb]{0.21960784,0.4627451,0.11372549}{Leafpad }}\textbf{cause it's
simple.\newline
Type in terminal:}\textit{ leafpad hello.c\newline
\ \ }  
\sitgfx[width=4.8693in,height=3.1602in]{reversing-img003.png}{fig:unk}{TODO CAPTION}
 \textbf{ }

\textbf{This opens a text editor for me to write my program in. So let's do that. Write it yourself, even if you are not
good at C I suggest you learn enough to do this so you understand what each line does. Your code should look like
this:}

\textbf{\newline
\ \ }  
\sitgfx[width=4.3854in,height=1.302in]{reversing-img004.png}{fig:unk}{TODO CAPTION}
 

\textbf{Now let's turn our code into a binary(which will be 32-bit, not 64, for reasons later explained).}

  
\sitgfx[width=6.5in,height=0.5555in]{reversing-img005.png}{fig:unk}{TODO CAPTION}
 \textbf{\newline
Listing (ls) the content in our file location now shows two hello files: \newline
{}-hello.c: our C source code\newline
{}-hello\_binary: our binary created from our source file\newline
\ \ }  
\sitgfx[width=4.7602in,height=0.6146in]{reversing-img006.png}{fig:unk}{TODO CAPTION}
 

\textbf{Running (./) hello\_binary achieved our objective, outputting the string we printed}

\textbf{What happens when the }\textbf{\textcolor[rgb]{0.21960784,0.4627451,0.11372549}{compiler}}\textbf{, a program
that interprets source code and outputs machine code, interprets our source code? Basically, A's and B's go through a
process and become 1's and 0's that make it possible for your processor (CPU) to understand. In fact, it's a little
less abstract than simply 1's and 0's. More on that later. }

\textbf{The compiler goes through your source, line by line, and tries to figure out what you meant. It turns your
source code into }\textbf{\textcolor[rgb]{0.21960784,0.4627451,0.11372549}{instructions}}\textbf{, or specific tasks
for your CPU to execute. You can't tell your CPU ``x = 5'' so your compiler of choice will tell it a series of
}\textbf{\textcolor[rgb]{0.21960784,0.4627451,0.11372549}{instructions }}\textbf{that represent ``x=5''.}


\subsection{We don't always get source code with our programs.}

\textbf{\newline
So what would happen if I deleted the source code for hello.c and renamed the binary? Would you have a clue what this
program even does until running it? And even then, your understanding of this program will be based on just the
behavior you can see when doing so. It is this fact that allows products like Chrome and Word to be shared to you
without making it easy for you to figure out EXACTLY how it works and sell your new product based off of this
information.  But there is still a way see what's inside for those who have the skill. Enter Reverse
Engineering...\newline
\newline
{}-How would we see how it works? The }\textbf{\textcolor[rgb]{0.21960784,0.4627451,0.11372549}{file }}\textbf{command
gives us some basic information about the file hello\_binary. ELF = Executable and Linkable Format. An .exe on Windows
is functionally equivalent to an ELF on Linux. The other thing to note on the output of
}\textbf{\textcolor[rgb]{0.21960784,0.4627451,0.11372549}{file}}\textbf{ is the phrase ``Intel 80386'' which is the CPU
}\textbf{\textcolor[rgb]{0.21960784,0.4627451,0.11372549}{architecture, }}\textbf{aka the physical hardware that this
program was compiled to run on. More specifically, 32-bit, which refers to the size of the
}\textbf{\textcolor[rgb]{0.21960784,0.4627451,0.11372549}{registers }}\textbf{of said CPU(more on those in a bit).}

\textbf{ }  
\sitgfx[width=6.5in,height=0.6252in]{reversing-img007.png}{fig:unk}{TODO CAPTION}
 

\textbf{So that's cool, but what else is in this program? The
}\textbf{\textcolor[rgb]{0.21960784,0.4627451,0.11372549}{strings }}\textbf{command lists all of the human-readable
strings in the binary of length 4 or greater. \newline
}  
\sitgfx[width=6.5in,height=3.5138in]{reversing-img008.png}{fig:unk}{TODO CAPTION}
 \textbf{\newline
We'll note that there are a lot of strings for such a tiny program! About 3 quarters of the way down, note our string
``hello world!'' from the source code. The compiler and version we used are a little bit below it. A lot of these
results are actually just to are left from the compilation process. We used the function ``printf'' which on linux is a
member of}\textbf{\textcolor[rgb]{0.21960784,0.4627451,0.11372549}{ glibc}}\textbf{ (the standard library for C
programs on Linux). So, the compiler had to bring in glibc in order to know what we meant by printf, hence all the
glibc/libc related strings.
}


\subsection{Programming and Compiling Hello World}

{\centering
\sitgfx[width=6.5in,height=1.778in]{reversing-img002.png}{fig:unk}{TODO CAPTION}
 \par}
\textbf{Here you can see my terminal open on the machine name(kali:\~{} root\#), kali, as user root, with \~{} symbol
letting me know I am in the home directory. Pressing Ctrl+L will clear the screen for me whenever I need.\newline
}

\textbf{Let's start by writing a simple ``hello world'' program in the C programming language. You can use any text
editor you like, but I'll use }\textbf{\textcolor[rgb]{0.21960784,0.4627451,0.11372549}{Leafpad }}\textbf{cause it's
simple.\newline
Type in terminal:}\textit{ leafpad hello.c\newline
\ \ }  
\sitgfx[width=4.8693in,height=3.1602in]{reversing-img003.png}{fig:unk}{TODO CAPTION}
 \textbf{ }

\textbf{This opens a text editor for me to write my program in. So let's do that. Write it yourself, even if you are not
good at C I suggest you learn enough to do this so you understand what each line does. Your code should look like
this:}

\textbf{\newline
\ \ }  
\sitgfx[width=4.3854in,height=1.302in]{reversing-img004.png}{fig:unk}{TODO CAPTION}
 

\textbf{Now let's turn our code into a binary(which will be 32-bit, not 64, for reasons later explained).}

  
\sitgfx[width=6.5in,height=0.5555in]{reversing-img005.png}{fig:unk}{TODO CAPTION}
 \textbf{\newline
Listing (ls) the content in our file location now shows two hello files: \newline
{}-hello.c: our C source code\newline
{}-hello\_binary: our binary created from our source file\newline
\ \ }  
\sitgfx[width=4.7602in,height=0.6146in]{reversing-img006.png}{fig:unk}{TODO CAPTION}
 

\textbf{Running (./) hello\_binary achieved our objective, outputting the string we printed}

\textbf{What happens when the }\textbf{\textcolor[rgb]{0.21960784,0.4627451,0.11372549}{compiler}}\textbf{, a program
that interprets source code and outputs machine code, interprets our source code? Basically, A's and B's go through a
process and become 1's and 0's that make it possible for your processor (CPU) to understand. In fact, it's a little
less abstract than simply 1's and 0's. More on that later. }

\textbf{The compiler goes through your source, line by line, and tries to figure out what you meant. It turns your
source code into }\textbf{\textcolor[rgb]{0.21960784,0.4627451,0.11372549}{instructions}}\textbf{, or specific tasks
for your CPU to execute. You can't tell your CPU ``x = 5'' so your compiler of choice will tell it a series of
}\textbf{\textcolor[rgb]{0.21960784,0.4627451,0.11372549}{instructions }}\textbf{that represent ``x=5''.}

\subsection{We don't always get source code with our programs.}

\textbf{\newline
So what would happen if I deleted the source code for hello.c and renamed the binary? Would you have a clue what this
program even does until running it? And even then, your understanding of this program will be based on just the
behavior you can see when doing so. It is this fact that allows products like Chrome and Word to be shared to you
without making it easy for you to figure out EXACTLY how it works and sell your new product based off of this
information.  But there is still a way see what's inside for those who have the skill. Enter Reverse
Engineering...\newline
\newline
{}-How would we see how it works? The }\textbf{\textcolor[rgb]{0.21960784,0.4627451,0.11372549}{file }}\textbf{command
gives us some basic information about the file hello\_binary. ELF = Executable and Linkable Format. An .exe on Windows
is functionally equivalent to an ELF on Linux. The other thing to note on the output of
}\textbf{\textcolor[rgb]{0.21960784,0.4627451,0.11372549}{file}}\textbf{ is the phrase ``Intel 80386'' which is the CPU
}\textbf{\textcolor[rgb]{0.21960784,0.4627451,0.11372549}{architecture, }}\textbf{aka the physical hardware that this
program was compiled to run on. More specifically, 32-bit, which refers to the size of the
}\textbf{\textcolor[rgb]{0.21960784,0.4627451,0.11372549}{registers }}\textbf{of said CPU(more on those in a bit).}

\textbf{ }  
\sitgfx[width=6.5in,height=0.6252in]{reversing-img007.png}{fig:unk}{TODO CAPTION}
 

\textbf{So that's cool, but what else is in this program? The
}\textbf{\textcolor[rgb]{0.21960784,0.4627451,0.11372549}{strings }}\textbf{command lists all of the human-readable
strings in the binary of length 4 or greater. \newline
}  
\sitgfx[width=6.5in,height=3.5138in]{reversing-img008.png}{fig:unk}{TODO CAPTION}
 \textbf{\newline
We'll note that there are a lot of strings for such a tiny program! About 3 quarters of the way down, note our string
``hello world!'' from the source code. The compiler and version we used are a little bit below it. A lot of these
results are actually just to are left from the compilation process. We used the function ``printf'' which on linux is a
member of}\textbf{\textcolor[rgb]{0.21960784,0.4627451,0.11372549}{ glibc}}\textbf{ (the standard library for C
programs on Linux). So, the compiler had to bring in glibc in order to know what we meant by printf, hence all the
glibc/libc related strings.
}

\section{Looking at the assembly of a binary}

\textbf{However, this is only scratching the surface of information we can extract from this binary. \newline
From here you start learning about the two core ways you can go about reverse engineering a program (Static
Analysis/Dynamic Analysis) and two of the most powerful tools we recommend you use for both (IDA Pro: Static, Radare2:
Dynamic). These 2 approaches and their associated tools are at the absolute core of Rev/Exploitation, which is why we
will be hitting these concepts in depth.\newline
\newline
\subsection{Static Analysis vs Dynamic Analysis}
In the simplest explanation I can think of, Static Analysis will show you all the information about your program that
can be seen without running it and Dynamic Analysis is everything you can learn about the code by running it and seeing
how it behaves. For me it helps to understand this using an example with high level code like C. \newline
\newline
If someone gave you the C source code to a program and told you to explain to them what it does or you will be fired,
how would you do that? Logically, I can think of two ways I would go about doing that. One way would be to open up the
c file, read it step by step and try to get a picture of what it does as a whole. The second way would be to compile
and run it, see what happens, how it behaves, and maybe connect some dots to get the gist of the program. Now already
we can tell that each one has its pros and cons. }

\textbf{For option one, the pro would be that we get to see everything the program might do; all the variables,
functions, and therefore, possible outcomes. You just have to put the work in and take the time to read every line of
that code. But what if the program is gigantic or if there are changes that happen to the program that will only happen
once you run it, like one that starts by decrypting itself. This is where option 2 would be necessary. }

{\centering
\textbf{\newline
For option two, running the program, we will not as easily be able to view all the inner workings of our program, but it
often shows us enough information to get a strong understanding in a much quicker time. \newline
}}

\subsection{Intro to Static Analysis}

\textbf{Let's open it in IDA Pro. Open IDA(Type into your terminal }cd \~{} \textbf{then }idademo70/ida )\textbf{ , then
select our binary when prompted. \newline
Once IDA finishes loading, it should look like this: }

{\centering   
\sitgfx[width=5.3957in,height=5.052in]{reversing-img009.png}{fig:unk}{TODO CAPTION}
 \par}
\textbf{What does all this crap mean?? At least we can see our string ``}Hello World!\textbf{{}''. \newline
We're going to take a break from the idea of hello world and dig into some
}\textbf{\textcolor[rgb]{0.07450981,0.30980393,0.36078432}{assembly.}}\textbf{ The words in the left column are the
names of }\textbf{\textcolor[rgb]{0.21960784,0.4627451,0.11372549}{instructions.}}\textbf{ For now, ignore everything
except the following block:}

{\centering
\textbf{\newline
\newline
 }  
\sitgfx[width=4.8126in,height=1.0209in]{reversing-img010.png}{fig:unk}{TODO CAPTION}
 
\par}

\textbf{Let's start at the ``call'' instruction and work backwards. You might be able to guess what the functionality of
this is after we google for what ``puts'' does.\newline
\newline
}\textit{Function: int puts (const char *s)}

\textit{The puts function writes the string s to the stream stdout followed by a newline. The terminating null character
of the string is not written. (Note that fputs does not write a newline as this function does.)}

\textit{puts is the most convenient function for printing simple messages. \newline
For example:puts ({\textquotedbl}This is a message.{\textquotedbl});}

\textit{outputs the text `This is a message.' followed by a newline.\newline
\newline
}\textbf{So, we can be pretty confident that ``}call puts'' \textbf{...calls puts. But puts takes an argument! Puts
takes a pointer to a char (this is how strings are denoted in C
}\url{https://en.wikipedia.org/wiki/Null-terminated_string}\textbf{ .) If we recall our programming 2 pointers, we know
that pointers point to memory locations(}\textbf{\textcolor[rgb]{0.21960784,0.4627451,0.11372549}{addresses}}\textbf{)
where the actual value can be found. \newline
The instruction on the same line as our string, }{}``lea''\textbf{ has the format lea operand1, operand2. ``LEA'' stands
for ``Load Effective }\textbf{\textcolor[rgb]{0.21960784,0.4627451,0.11372549}{Address}}\textbf{{}'' and puts the
}\textbf{\textcolor[rgb]{0.21960784,0.4627451,0.11372549}{address}}\textbf{ of operand2 into operand1. (Nearly every
x86 instruction has the format }instruction operand1, operand2. \textbf{Usually it's referred to as }destination,
source\textbf{ instead of }operand1, operand2.)\textbf{  \newline
So with all of this in mind, we can be fairly confident that a pointer to the string ``Hello World!'' is getting stored
in operand1, which is }\textbf{\textcolor[rgb]{0.21960784,0.4627451,0.11372549}{edx}}\textbf{.
}\textbf{\textcolor[rgb]{0.21960784,0.4627451,0.11372549}{Edx }}\textbf{is a
}\textbf{\textcolor[rgb]{0.21960784,0.4627451,0.11372549}{register}}\textbf{. The CPU has a lot of registers, and
registers are small bits of fast storage located on the physical CPU. If you remember from earlier, this program is a
32-bit executable which runs on a CPU with 32-bit registers! Nearly every CPU instruction has a register involved.}

{\centering
  
\sitgfx[width=1.9583in,height=1.8752in]{reversing-img011.png}{fig:unk}{TODO CAPTION}
 \par}
\textbf{So we know that a pointer to (the address of) our string gets put into edx and that puts gets called. And when
we run the program, we know that ``hello world!'' gets printed. That encapsulates the functionality of our hello world
program. }

\subsection{Analyzing logmein lvl1(aka log)}

\textbf{Now that we're a little familiar with the basics, let's look at something a little
}\textbf{\textit{harder}}\textbf{. This program came from the wild (}\textbf{\textcolor{red}{or a CTF maybe)}}\textbf{
and we don't know anything about it besides the name - log. What can we do to figure it out?From your terminal, cd to
Desktop/rev\_be101/binaries by issuing the command }cd Desktop/rev\_be101/binaries\textbf{.
}\textbf{\textcolor[rgb]{0.21960784,0.4627451,0.11372549}{File }}\textbf{says its a 32-bit executable for linux, and
}\textbf{\textcolor[rgb]{0.21960784,0.4627451,0.11372549}{strings }}\textbf{has some interesting output. Do these on
your own. \newline
\newline
Now let's open it in IDA. Type into your terminal }cd \~{} \textbf{then }idademo70/ida -\textbf{you can use the Tab key
to autocomplete. Once IDA gets open, agree to the license agreement and then choose to open a new file. Navigate to
Desktop/rev\_be101/binaries and open our file }log. \textbf{We'll land in }\textbf{\textit{main }}\textbf{ {}- if for
some reason you navigate away from this screen, double click ``main'' on the functions screen. If your graph view turns
funky, try pressing the spacebar to get back to the graph.}  
% Unhandled or unsupported graphics:
\sitgfx[width=6.5in,height=3.2638in]{reversing-img012.png}{fig:unk}{TODO CAPTION}
 

\textbf{Let's scroll down a bit with the arrow keys or mouse drag (about 2 boxes, to the circled portion) until we see
this:  }

  
\sitgfx[width=6.5in,height=3.4862in]{reversing-img013.png}{fig:unk}{TODO CAPTION}
 

\textbf{From this one screenshot we }\textbf{\textit{can}}\textbf{ see pretty much the entire functionality of this
program, without even having had to run it!!! That is, if we look closely and can read assembly. Let's direct our focus
to the ``call'' instructions. If you click once on the word ``call'' IDA will highlight all of the appearances for you.
Take a look at your own IDA and think about what this all represents. It's okay if you're mostly lost! Once again,
keeping up with the theme of }\textbf{\textit{reverse}}\textbf{, we're going to be working backwards. \newline
\newline
We can see a few ``call'' instructions. The first code block should look somewhat familiar from the hello\_world
example, except instead of edx and puts, it's eax and printf. Go back and look at that again if you don't remember what
that looked like.}

  
\sitgfx[width=6.5in,height=4.611in]{reversing-img014.png}{fig:unk}{TODO CAPTION}
 \textbf{\newline
The second call is to scanf, another familiar function from C. We see that \%s gets put into eax and pushed before scanf
gets called. Before that, a }\textbf{\textcolor[rgb]{0.21960784,0.4627451,0.11372549}{local variable
}}\textbf{(var\_30) gets put into eax and pushed. Notice how ``push eax'' happens right before(sometimes multiple
times) each call in the above screenshot? The address of a string gets put into eax and then pushed before printf, then
the addresses of a local variable and another string (the format string ``\%s'') gets put into eax and then pushed, and
finally the addresses of two }\textbf{\textcolor[rgb]{0.21960784,0.4627451,0.11372549}{local variables }}\textbf{get
put into eax before getting pushed before check\_password gets called. If you guessed that these values that get pushed
are the arguments for these functions, you'd be right. And we know all without ever running the program! Let's go ahead
and run the program now with our newfound knowledge and see what happens. (ctrl+c exits the currently running
program)\newline
}  
\sitgfx[width=6.5in,height=1.139in]{reversing-img015.png}{fig:unk}{TODO CAPTION}
 

\textbf{Do you see where printf and scanf come into play? And can you see where check\_password happens? Please go back
to IDA. At the end of the code block with the ``incorrect. Try again:'' message, there is an arrow that goes back up to
the block containing check\_password(). This is indicative of a loop, as we saw when we ran the program. The program
prompts for input, checks it, and if it's wrong, says ``Incorrect. Try Again:'' and waits for more input.  }
\sitgfx[width=6.5in,height=3.028in]{reversing-img016.png}{fig:unk}{TODO CAPTION}
 \textbf{\newline
Feel free to keep running the binary and trying to guess the password. We're going to find it ourselves. But first, some
more assembly. \newline
}

\clearpage{\centering
\textbf{Let's take a look at the last two instructions of the block containing check\_password. There's }\textit{test
eax,eax}\textbf{\textit{ }}\textbf{and} \textit{jz short loc\_5C1}\textbf{. The }\textbf{\textit{test
}}\textbf{instruction (for our purposes) checks if operand1 + operand2=0 by performing a bitwise AND on operand1 and
operand2. If they are, it sets the Zero Flag to 1. Otherwise, it sets the Zero Flag to 0. The Zero Flag is part of the
EFLAGS register, specifically the ZF bit. So the next instruction, jz(jump if Zero Flag is set to 1), checks the Zero
Flag, and if it's set to 1, it takes the green path. If it's set to 0, it takes the red path. Because the last function
called before these instructions was check\_password, we can assume that check\_password stores its result in eax. So
if check\_password returns 1, we take the red path(win). If it returns 0, we take the green path(lose/try again). Even
though we haven't even figured out how to win, we know so much about the program! \newline
}  
\sitgfx[width=6.4583in,height=4.5728in]{reversing-img017.png}{fig:unk}{TODO CAPTION}
 \textbf{\newline
Now let's get into the meat of the program. Double-click on the word check\_password on the graph, or double click on it
from the functions window (it's right below main). Take a minute or two to familiarize yourself with the structure of
the function, and recall everything we know about what happens with this function. We just figured out that if
check\_password returns 0, we lose and try again, and if it returns 1, we win! Let's start at the very end of the
function, the win and lose conditions, and work our way back. }  
% Unhandled or unsupported graphics:
\sitgfx[width=6.5in,height=3.8055in]{reversing-img018.png}{fig:unk}{TODO CAPTION}
 
\par}

\textbf{Let's look at the code block circled. We know that the function check\_password returns its value in eax, since
the }\textbf{\textit{test}}\textbf{ instruction from before was }\textbf{\textit{test}}\textbf{ing eax,eax. We get more
proof of that now, where eax is the last register to be modified before the retn (return) instruction. It appears we've
found the }\textbf{\textit{lose/try again}}\textbf{ condition! We can see that the top block circled ``mov''{}'s(the
}\textbf{\textcolor[rgb]{0.21960784,0.4627451,0.11372549}{mov}}\textbf{ instruction copies operand2 into operand1) 0
into [ebp+var\_8] which is the location of a local variable. The next block of code }mov's\textbf{ that same
[ebp+var\_8] into eax, before returning. So, we can safely assume that [ebp+var\_8] is the location of the return
value.\newline
}(In Intel x86 assembly notation, the brackets [] denote the memory contents at a specific address. This is important
because \textit{ebp} stands for \textit{base pointer} - it points to the \textit{base address} of The
\textcolor[rgb]{0.21960784,0.4627451,0.11372549}{Stack}. So [ebp +var\_8] is the compiler's way of telling the CPU
``[go to this memory address and get whatever is there.]'')

\textbf{So now we know how to lose. Can we find how to win? If we continue working backwards from the return point, we
see that red arrow that also points to the return block. Double click it or follow it back up.}

{\centering   
\sitgfx[width=3.1146in,height=4.4272in]{reversing-img019.png}{fig:unk}{TODO CAPTION}
 \par}
\textbf{The light blue arrow is the same one previously mentioned. Could it be the win condition?? Let's take a closer
look at the code block above it.}

{\centering   
\sitgfx[width=4.6354in,height=1.5in]{reversing-img020.png}{fig:unk}{TODO CAPTION}
\par}
\textbf{We see some old friends, and some new ones. This }\textbf{\textit{test}}\textbf{ instruction looks kind of
familiar but also kind of wrong! }  
\sitgfx[width=6.5in,height=4.8752in]{reversing-img021.png}{fig:unk}{TODO CAPTION}
 

\textbf{It appears AL is the lowest 8 bits of the EAX register. So }\textit{test} al, al\textbf{ is really
}\textit{test} the lowest 8 bits of eax, the lowest 8 bits of eax\textbf{! But what could be in eax at this point in
the program? Do you have any guesses? Recall that when we ran the program on our command line, it asked for input and
told us that we were wrong. We know how to be wrong, but how can we be right? This function compares our input string
to another that is located somewhere in the program. We could continue statically analyzing the function and we would
find the location of the victory string, but we would never be able to determine exactly what the password is. For
this, we turn to dynamic analysis. We'll be using a tool called radare2, a combined disassembler and debugger. } 

\sitgfx[width=6.5in,height=4.3335in]{reversing-img022.png}{fig:unk}{TODO CAPTION}
 \textbf{To open this file in debug mode in radare2, type `}r2 -d log'\textbf{, then after it gets open, `}aaa'\textbf{
and then `}afl\textbf{{}'. Aaa tells r2 to analyze all functions, and afl tells it to list the analyzed functions.}

  
\sitgfx[width=6.5in,height=4.1807in]{reversing-img023.png}{fig:unk}{TODO CAPTION}
 

  
\sitgfx[width=6.5in,height=4.9028in]{reversing-img024.png}{fig:unk}{TODO CAPTION}
 \textbf{\newline
We see some old and tired friends! Type `}s main' \textbf{to instruct r2 to seek to main, then `}db main' \textbf{and
`}dc\textbf{{}' to instruct radare to set a breakpoint at main and then continue until the breakpoint.}

\textbf{Now type ``V'' then enter to open visual mode}

  
\sitgfx[width=6.5in,height=4.2362in]{reversing-img025.png}{fig:unk}{TODO CAPTION}
 

\textbf{The first visual mode is a basic hexdump of the binary. This can be useful at times but we will go for a
different view for now. type ``p'' twice without hitting enter. Anything look familiar? It's the same disassembly from
IDA, just now in radare's format with some bonus info.}

  
\sitgfx[width=6.5in,height=4.7362in]{reversing-img026.png}{fig:unk}{TODO CAPTION}
 

\textbf{This may be a little bit to take in if this is completely new to you so take a deep breath, but don't worry.
It's not so bad. Now, you can see a view of the instructions in memory that are going to be executed at the bottom of
the display. The rows of 0x00000000's with labels are the
}\textbf{\textcolor[rgb]{0.21960784,0.4627451,0.11372549}{registers }}\textbf{of the computer and at the top of the
screen is a view of the }\textbf{\textcolor[rgb]{0.21960784,0.4627451,0.11372549}{stack}}\textbf{.}

  
\sitgfx[width=5.8335in,height=3.6457in]{reversing-img027.png}{fig:unk}{TODO CAPTION}
\textcolor{red}{***** Don't know much about how a Stack works? Read below. If you are well informed, skip
this*****}\newline
\textbf{The Stack}

\textbf{The stack is a programming structure in memory that works as a first in, last out queue. Items as ``pushed'' on
the stack in a certain order and then ``popped'' off in the reverse order. So, for example, say my stack starts out
with a 1.}

\textbf{0x0010: 1}

\textbf{and then I push a two(2) and then a three(3) onto the stack \newline
in that order - push 2 \newline
\ \   {}- push 3}

\textbf{0x0008: 3\newline
0x0009: 2\newline
0x0010: 1}

\textbf{Now If I were to pop something off the stack, I would first get a three (since it is on top of the stack) and
the stack would then look like}

\textbf{0x0009: 2\newline
0x0010: 1}

\textbf{Inside of your computer, the stack works the same way. The numbers pushed and popped off the stack of a 64 bit
computer can be variables, memory addresses, and other data and will look more like 0x00007f11e8a9bf2a (a memory
address) or 0x656b6f4456964f (A piece of inputted user data) This process is referred to as changing the stack frame.}

\textbf{One caveat of the stack is that in computer memory it grows upward toward }\textbf{\textit{lower memory
addresses}}\textbf{ so when you push numbers onto the stack the value of the topmost data on the stack is at the lowest
memory address:}

\textbf{0x0008: Topmost value on stack is at lowest address (0x0008)\newline
0x0009:\newline
0x0010: lowest value of stack is at the highest address (0x0010)}

\textcolor{red}{******************END OPTIONAL STACK EXPLANATION HERE****************************}

\textbf{For those new to assembly, registers are like variables for assembly. You can move data into and out of
registers (mov eax, 6)and use them to push and pop data to and from the stack (pop rax). Several registers have special
purposes. }\textbf{\textcolor[rgb]{0.21960784,0.4627451,0.11372549}{eip}}\textbf{ is used to point to the next assembly
instruction the computer is going to execute in memory. Right now, assuming you typed }s main\textbf{, }db
main\textbf{,and }dc\textbf{ above, }\textbf{\textcolor[rgb]{0.21960784,0.4627451,0.11372549}{eip}}\textbf{
is0x565d054d.}\textbf{\textcolor{red}{ }}\textbf{(The IP of EIP stands for Instruction Pointer.)}

\textbf{\textcolor{red}{ }}\textbf{If we look at the code at the bottom panel we see this agrees with the topmost
instruction in memory.}

  
\sitgfx[width=6.5in,height=5.6665in]{reversing-img028.png}{fig:unk}{TODO CAPTION}
 

\textbf{The register esp is used to point to the top of the stack. As you can see, the address in esp is 0xffd3008c
which agrees with the highest stack address in the top view.}

  
\sitgfx[width=6.5in,height=5.1252in]{reversing-img029.png}{fig:unk}{TODO CAPTION}
 

\textbf{main is the execution starting point of programs written in C and C++ for command line. the ``db'' command sets
a breakpoint inside the code of the program. So by typing ``db main'' you set a breakpoint at the beginning of the main
function, which is the start of the program. (Radare2 actually starts a program and breaks before the main function or
the program starts.) Other functions can be used the same way. For example if a program had a function in it called
sayHello() then ``db sayHello'' would put a breakpoint at the beginning of that function. You can enter offsets into
the ``db'' command as well. You can type ``db main+1'' to set a breakpoint at the second instruction of the main
function. The ``dc'' command stands for debug continue, and runs the program until another breakpoint is hit.
Breakpoints can also be set by using the up and down keys(in visual mode) to look through memory and then hitting shift
+ b to place a breakpoint at the highest instruction in the display. (You cannot use this when the tab for command
execution is open at the bottom of the screen. hitting enter without typing any command will get rid of the tab.)
Hitting shift + b again will remove the breakpoint. or using a minus (-) in the ``db'' command. For example ``db
-main'' will remove the breakpoint at the start of main.}

\textbf{Press the `}:\textbf{{}' character while in Visual Mode to bring up the command entry. Let's set a breakpoint on
the call to check\_password by bringing up command entry with :, then typing }db sym.check\_password \textbf{(you can
use tab to autocomplete here as well).  If you run ``dc'' now, you will see the execution of the program. First,
``Please enter a word or character:'' is printed to the screen}  
% Unhandled or unsupported graphics:
\sitgfx[width=6.5in,height=5.3335in]{reversing-img030.png}{fig:unk}{TODO CAPTION}
 

\textbf{I gave it the word ``asdf''. I pressed enter after typing asdf just like when we ran the program before, and I
get the message hit breakpoint at: 565d0625. Press enter to return to visual mode. Did you notice the change? Instead
of being in }main\textbf{ we're now in }sym.check\_password\textbf{! }  
% Unhandled or unsupported graphics:
\sitgfx[width=6.5in,height=5.2083in]{reversing-img031.png}{fig:unk}{TODO CAPTION}
 

\textbf{There are some other things to note here as well. In the stack view(top block) , at the very end of the visible
stack, we see our input string! Asdf! ...and there's another suspicious string in the stack view too. Let's set a
breakpoint after the program initializes its local variables (}:, \textbf{then }db 0x5664a63c, \textbf{then press
enter- pro tip- select the memory address with your mouse and middle click it to copy and paste in one click!). Then,
press }: \textbf{and type }dc \textbf{then press enter to continue to the next breakpoint. R2 should tell you ``hit
breakpoint at }0x5664a63c.\textbf{{}'' }  
\sitgfx[width=6.5in,height=5in]{reversing-img032.png}{fig:unk}{TODO CAPTION}
 

\textbf{Let's get debugging. Press ``s'' to single step through the program. You will notice the highlighted memory
address will move down to the jmp instruction. Press ``s'' again and we will end up in a block of familiar code!} 

\sitgfx[width=6.5in,height=3.778in]{reversing-img033.png}{fig:unk}{TODO CAPTION}
 

\textbf{Let's take a step away from the flow of the program to examine the values in the registers. Before we execute
any of this block, we see that there are some values currently stored in eax and edx, but  they don't really have a lot
of meaning to us at this moment. Press ``s'' twice so that the highlighted instruction is: \newline
}0x565cf672  \ \ 01d0  \ \ add eax, edx\newline
  
\sitgfx[width=6.5in,height=1.4445in]{reversing-img034.png}{fig:unk}{TODO CAPTION}
 

\textbf{We see that both edx and eax have new values! Edx got 0x000000000 (aka 0) and eax got 0xffb57da8! What the heck
is 0xffb56da8?? It looks kind of like the esp..but remember in hexadecimal we count 8,9,a,b,c,d,e,f. And the base
pointer (the bottom of the stack) is 0xffb57d}\textbf{\textcolor{red}{68}}\textbf{...so this memory address is below
the stack for this function. \newline
We can get the data at this memory address by pressing `:' and issuing the command ``}pxw 40 @ eax'' \textbf{or ``}pxw
40 @0xffb57da8''\textbf{. They're the same thing, since eax = 0xffb57da8. This is similar to how the brackets [] work
in x86 assembly. When we type }pxw \_nbytes\_ @ memory \textbf{, it's just like saying ``get nbytes from [memory].''We
see our input string that we entered when the program was in main!!!!  Let's keep going.}\newline
   
\sitgfx[width=6.5in,height=1.389in]{reversing-img035.png}{fig:unk}{TODO CAPTION}
 

\textbf{Press ``s'' 2 more times so that the highlighted instruction is  \ \ }

0x565cf677\ \ 84c0  \ \ test al, al

  
\sitgfx[width=6.5in,height=2.5138in]{reversing-img036.png}{fig:unk}{TODO CAPTION}
\textbf{The first time we pressed s just now, nothing changed. It added edx to eax, which was adding 0 to a number, no
effect. The next instruction is more interesting. Wise persons might recognize, even without knowing what the movzx
instruction does, that eax contains the hex representation of an ASCII value. Some quick googling reveals that eax now
contains the ASCII representation of the letter ``a'', Hx 61 in the chart. We even reverse engineered the functionality
of }  
\sitgfx[width=6.3543in,height=0.2291in]{reversing-img037.png}{fig:unk}{TODO CAPTION}
 !\newline
\textbf{We figured out that it moves 1 byte (the size of a char in C) from operand2 into operand1.  More specifically,
in this instance it moves a byte from the address that eax currently holds ([}0xffb57da8] = asdf = 0x66647361\textbf{)
and puts it into eax. Let's keep going. Press ``s'' through the test and jump instructions(2x) ,and then 3 more times
so we're looking at this screen.}  
\sitgfx[width=6.5in,height=1.8335in]{reversing-img038.png}{fig:unk}{TODO CAPTION}
 

\begin{center}
 
  \sitgfx[width=1.6043in,height=4.9272in]{reversing-img039.png}{fig:unk}{TODO CAPTION}

\end{center}
\textbf{We see the same functionality from the last block we analyzed, albeit with some more instructions. This time, it
movzx's a byte (char) from our input string into edx, instead of eax. Step through this instruction and verify for
yourself.\newline
Cool. Let's keep stepping to try to figure out what happens next. It moves a local variable, the same ones from before,
4h, which we know is 0, into ecx. Verify that as well. The next instruction is juicy. One of the function arguments
gets moved into eax! Let's do} : \textbf{then }pxw 50 @ eax.   
% Unhandled or unsupported graphics:
\sitgfx[width=6.5in,height=4.5138in]{reversing-img040.png}{fig:unk}{TODO CAPTION}
 

\begin{center}
 
  \sitgfx[width=1.8543in,height=0.2189in]{reversing-img041.png}{fig:unk}{TODO CAPTION}

\end{center}
\textbf{EAX now holds the memory address of (a pointer to) a string that we did not enter!! So EDX holds the address of
our input string and EAX holds a foreign string. What happens next?After a useless }add eax,ecx\textbf{, our new friend
movzx moves a byte(character) from the string pointed to by eax into eax! }  
% Unhandled or unsupported graphics:
\sitgfx[width=6.5in,height=2.0555in]{reversing-img042.png}{fig:unk}{TODO CAPTION}
 

\textbf{So now we have our 0x61 (lowercase A) in edx, and 0x55 (capital U) in eax. Can you guess what the next
instruction, }cmp dl, al \textbf{does? If your answer was }  
% Unhandled or unsupported graphics:
\sitgfx[width=6.5in,height=1.8055in]{reversing-img043.png}{fig:unk}{TODO CAPTION}
 

\textbf{You'd be correct! The cmp instruction sets the Zero Flag to 1 only if the two operands are equal. So, if the
character of the input string, `a' and the character of the password string, U, are the same, then we get to take the
jump. Feel free to keep stepping through the program and repeating our commands, but I think we've done enough
reversing on this binary. Let's run it again, this time giving it what we think the password is. In r2, press q to
return to the r2 console and then type qy to quit radare2. Run log (./log) and give it the password!! }
\sitgfx[width=6.4165in,height=1.3126in]{reversing-img044.png}{fig:unk}{TODO CAPTION}
\url{https://www.youtube.com/watch?v=5pZa8YobAeg}

\section{Reverse Engineering -- Static Analysis Walkthrough}

\textbf{Static analysis is done to help you understand how the program should function before running it. The purpose of
static analysis is to use the program flow and the assembly code to understand how the program should run and how to
find out which sections are important and should be further analyzed. This is important because some programs use many
jumps and blocks of code, and it is important to see how the program flows before jumping into debugging. On top of
this, static analysis does not require a run-time environment, so it can be done from any machine. I use my faster Mac
host instead of my boxes, but any operating system will work!}

\textbf{To start, we will be statically analyzing a binary using a tool called IDA. IDA is seen as an industry standard
for static analysis because it has tons of useful plugins, many different features, and even the ability to script with
Python. However, this is why the professional version costs a few hundred thousand dollars. We will only be using some
of the more simple and useful features for this tutorial, and all of these can be found in the free IDA demo that is
provided.}

\textbf{The program that we are analyzing is called
}\textbf{\textcolor[rgb]{0.21960784,0.4627451,0.11372549}{example}}\textbf{, and we plan on discovering what this
program does. It was written in C and compiled with GCC with the m32 flag, for compatibility on 32 bit machines. Rather
than showing you the C code that was used, we will use the program flow to determine the C code from the assembly
language.}

\textbf{To open a file in IDA, we will launch IDA. You will see a screen asking whether you would like to disassemble a
new file, work on your own, or load a previous disassembly. This is your first time disassembling this file, so you
will click the box that says ``New''. It will then present a box of options that affirms the format of the 32 bit ELF
file. Make sure that IDA will auto-comment, and select ``OK'' to proceed. After some loading as IDA evaluates the
program, you should get a screen that looks like the following:}

  
\sitgfx[width=6.5in,height=4.0626in]{reversing-img045.png}{fig:unk}{TODO CAPTION}
 \textbf{This is the graph mode of IDA, and it presents the functions as series of connected boxes presented as jumps.
The other main view of IDA is known as the text view, and it presents all of the information as it is loaded into RAM.
This is more confusing to read and involves the format of the ELF file and all of it's sections, so we will start by
using Graph view. If you are not already in this view, press the spacebar to switch between these two main views.}

\textbf{In this view, there are a few key features to look at for this program. On the left, there is an area for the
functions, known as the ``functions window''. In this, all of the functions that are used in the binary can be seen.
However, the majority of them are functions used by GCC and by the stdio.h file that is imported. This can be
determined by the functions with the purple shade, like \_strcmp, \_gets, \_puts, \_strlen, and libc\_start\_main.}

\textbf{It is important to determine which functions are used because they can indicate how the program works. In this
case, we can guess that two strings are compared due to the strcmp method, and that we will give input with gets and
see output with puts. The strlen could be used to find the length of a string that we enter or a string that is
hardcoded in memory. All of the other functions that are used take a part in making sure that the ELF can run
independently, and they help link all of the parts of the file and libraries together in virtual memory. This is a
complicated process, but luckily these functions don't affect the flow of the program that we can affect.}

\textbf{On top of this, it is always helpful to look at the strings that can be found in the program before we analyze
further. IDA can look for areas in memory in which a series of consecutive memory falls into the printable range of
ASCII. In other words, it looks through the whole file for strings of words and numbers that may end up being printed
on the screen or used in various functions. To open this view, click on ``Views'' in the toolbar, then ``Open
subviews'' and ``Strings''. The keyboard shortcut for the strings subview is Shift-F12. Once you open the strings
subview, you should see the following screen:}

  
\sitgfx[width=6.1252in,height=3.8335in]{reversing-img046.png}{fig:unk}{TODO CAPTION}
 \textbf{Most of these strings are function names, and they are linked to the libc functions that we import in the
stdio.h file that I previously mentioned. These all have an address that begins with LOAD, as can be seen. This means
that they are used when the program loads libc and imports the functions and references that will be used. After these,
there are 5 strings that stand out. The last string is stored in the eh\_frame, as can be seen in its address, and it
comes from the exception handling part of the ELF file, so it is not relevant in the execution that we can see. By now,
you can probably see that ELF files and other executables are way more complicated than they seem in the C code that is
used to create them, and IDA shows all of the sections used by the compiler to make sure that this code can run in the
computer and load into its RAM properly. It's a very intricate and entertaining process, but it is usually not too
important when reverse engineering. I would not recommend affecting any of this data because it could destroy the
program, for all practical purposes.}

\textbf{Now let's look at the 5 strings that seem interesting. The first says, ``If you managed to get here, you used an
exploit!'' That sounds like a place in the program that is not meant to be reached and can only be reached with an
exploit, but we are not focused on exploitation yet. I will reference that later in this tutorial, after understanding
the basic flow and how to reverse engineer the program in the way that it is supposed to function. The second string
says, ``Please enter the password:''. This seems like a prompt that would be printed before we give input. The third
strings says, ``itisnotthiseasy''. That one seems confusing, but it may be useful. The fourth string says, ``Correct!''
We can assume that this is the part of the program that we would like to reach, because you may assume that this
relates to the password that it asks us to enter. The fifth string seems like the other possibility, as it states,
``Mission failed, we'll get them next time''. You may have noticed that all of these have addresses that begin with
``.rodata'', which stands for ``read only data.'' This means that these strings can only be read by the program, and
they are located in an area of memory that cannot be written to or executed. Now, it is time to check out the flow!}

\textbf{Click on ``IDA View-A'' in the list of tabs to go back to the graph view. Now we will read through the graph and
attempt to understand what the program should do. Starting from the top box, a few definitions are created for you by
IDA. The local variables and labelled ``var\_10'' and ``var\_C''. These variables are created and used inside of this
function. They are stored in lower memory addresses than EBP, so IDA can recognize that a repeated negative offset from
EBP indicates a local variable in memory. The arguments are labelled ``argc'', ``argv'', and ``envp'', and they have
positive offsets from EBP. You may recognize these because they are commonly referred to in C programs where the main
function has command line arguments. However, you can look through the graph to find out where any of these arguments
are referenced, and it can be seen that they are only declared and never used. On top of this, ``s'' is defined, and
IDA commonly uses the offset ``s'' to refer to strings. In the first block everything up to ``lea'' is used to set up
the stack and position the memory, so this can be better analyzed in dynamic analysis. }  
% Unhandled or unsupported graphics:
\sitgfx[width=6.4583in,height=2.4791in]{reversing-img047.png}{fig:unk}{TODO CAPTION}
 \textbf{Once we reach the lea, knowledge of x86 assembly language becomes incredibly important. I always try to keep a
reference manual at hand in case I do not recognize an instruction, but Google often turns out to be the biggest help.
Lea stands for ``load effective address'', so it pretty much functions as a way to store a pointer. In this case, the
pointer for some offset of register ebx is stored in register eax, and luckily IDA simplified life by displaying the
string that is at that section of data. The pointer to this string is stored in register eax, and the next instruction
``push''es this pointer onto the stack. After this, the ``puts'' call is performed, in which the puts function of the
stdio.h library is called. The libc manual says that ``puts'' takes one argument, a string pointer, and it prints this
memory to standard out, or the screen. IDA recognized that eax was on top of the stack and commented that the value in
eax is the argument passed to the puts call. Any blue phrase that begins with a semicolon in IDA is an indication of an
argument for the next mentioned function call. We have already determined that eax held the pointer the string ``Please
enter the password:'', so this string must be printed!}

\textbf{ \ \ After this, 10h (10 in hex) is added to esp to move the stack to clear the frame created for the puts call,
and then 0Ch is subtracted to make room for the next call. Don't worry about not understanding how the stack works too
much in this tutorial because it is much easier to visualize in dynamic analysis, and it is not worth stressing about
or getting caught up on. However, it is helpful to change the 10h and 0Ch into decimal to be more easily read. To do
this, select the green text for either number and type ``H''. This will convert from hexadecimal to decimal format, and
it is now much easier to read. Now, let's look at the next function call. The address of ebp+s is loaded into eax, and
then eax is pushed. After this, gets is called. We know that gets uses one argument, which is the pointer to the area
in memory in which the string is to be stored. Now we know that location ``s'' is where our string that we input is
stored. This must be out ``password''!}

\textbf{ \ \ For the next function call, the address of our password is loaded again into eax and pushed before strlen
is called. This must determine the length of the string that we input. After the stack pointer is added again, a mov
instruction is used. This instruction moves the value stored in one register or area of memory into another. IDA uses
Intel syntax, so you read it from right to left. The value of eax is moved into ebp+var\_10, and eax is the standard
register for return value. This means that var\_10 is our string length! Let's rename it! To do this in IDA, click on
the green ``var\_10'' and type n. We will name it input\_length. Once you do this, you will realize that all occurences
of ``var\_10'' have been replaced with ``input\_length''. 0 is moved into var\_C, so we do not know what it's function
is yet. After this, jmp is used to reference an unconditional jump, and we leave the box. Boxes in IDA end with jumps
or endpoints.}

\textbf{ \ \ In the next box, var\_C is moved into eax, and this is compared with input length, and if it is less, the
green jump is taken. Decisions are being made, which seems scary! But don't worry, there are only two options for the
flow at this time. The first time, unless no input is given, 0 must be less than the size. This means we will follow
the green arrow to the next box. }  
\sitgfx[width=2.5937in,height=2.3335in]{reversing-img048.png}{fig:unk}{TODO CAPTION}
 

\textbf{We will approach this section like we did with the previous boxes. At first, the pointer of the string is loaded
into edx, and var\_C is moved into eax. These are added together and stored in eax (remember, we read from right to
left for Intel and IDA uses Intel syntax). Now let's look at all that we did so far and try to represent it in C code:
var\_C = var\_C + s*. The next instruction looks ugly, but we can tell that it involves moving. The byte referenced to
by the pointer at eax is moved into eax, and an Intel manual shows that the upper 3 bytes of register eax are zeroed.
Then, 1 is added to eax, and the contents of register eax are moved into ecx. This is confusion, but we will go back to
this and why it happens. Then, the address of string s is loaded into edx and var\_C is moved into eax again, and they
are added together and stored into eax again. After this, register cl (the least significant byte of register ecx) is
moved into the memory referenced to by eax, and then var\_C is incremented.}

\textbf{Hmm, if var\_C is incremented each time, starting at 0 and ending at the length of the string, this seems like a
for loop. Indeed, in the C code this was a for loop! We can rename var\_C ``counter'' to simplify our reading. To
understand what the box we just read does, it helps to know that a character is one byte in size. From the first half
and the C code that we created, we can tell that the pointer to each character consecutively in the loop is loaded into
eax, and then a byte at that location is moved to eax. This must be the characters, from the first character we enter
to the last! Then, 1 is added to this value, so the character at that value is incremented. After this, the code
repeats and our C code line that we created is used again to reference the location of the character. However, the
character in ecx is moved back into memory. This means that the characters are each incremented, starting with the
first and ending with the last! It may seem tricky at first, but the ability to recognize loops and recognize the
traversal of arrays becomes more natural with time. If you are confused you should read again after knowing the
outcome, it may make more sense.}

\textbf{Now that we have traversed the word, we can follow the red arrow.}

  
\sitgfx[width=6.4898in,height=3.948in]{reversing-img049.png}{fig:unk}{TODO CAPTION}
 \textbf{Firstly, the pointer to the string ``itsnotthiseasy'' is loaded into eax, which is then pushed onto the stack.
This is referenced to as argument s2. The string in memory that we entered and that had each character incremented by 1
is then loaded into eax and pushed onto the stack as argument s1, and strcmp is called. This compares the two strings,
returning 0 if they are equal and other numbers if they are not. Test eax, eax tests to see if the value in eax is zero
or not. If it is zero, it sets a special zero flag. If this zero flag is not set, the green arrow is taken, and the
string ``Mission failed, we'll get them next time.'' is printed. This is a long string, so you can double click
``aMissionFailedW'' to go to that section of memory and read the whole string.}  
% Unhandled or unsupported graphics:
\sitgfx[width=6.5in,height=0.6563in]{reversing-img050.png}{fig:unk}{TODO CAPTION}
 \textbf{27h is used to refer to the single quote because that is it's ASCII value and it is a special character in this
case. This allows it to be used in the string even if it is used to denote the start or stop of a string here. To go
back to the previous part, click on the back facing arrow above the function list.}

  
\sitgfx[width=4.1252in,height=2.1772in]{reversing-img051.png}{fig:unk}{TODO CAPTION}
 \textbf{\newline
This example of stepping backwards in our analysis can also be seen as a usage of a cross reference. A cross reference
is a location in the program in which this function or string is called and used, and it can be a useful way of finding
when functions and strings are used in the flow of the program. To find the cross references to this string, you can
right click on it and select ``List cross references to.'' This can also be done with the keyboard shortcut ``ctrl-x''.
When you do this, you should see the following:}

  
\sitgfx[width=6.4898in,height=1.6043in]{reversing-img052.png}{fig:unk}{TODO CAPTION}
 

\textbf{This is the location in the graph in which we found this string, because it was the code that loaded the pointer
to the string into register eax! Double click this to go back to the graph view of our function, and we can continue
from there.}

\textbf{ \ \ The final block helps end program execution by loading 0 into eax and popping all of the variables off of
the stack. The 0 in eax is a standard in programming to represent no errors in execution, and it is used because in C
programs, the main method is forced to return an integer as it is declared ``int main()''. We have statically analyzed
the whole program!}

\textbf{ \ \ Now, let's look at the bigger picture. We give a password to the program, every character is incremented by
1, and then it is compared to the string ``itsnotthiseasy.'' To do this successfully, we should decrement all of the
characters of ``itsnotthiseasy'' to get the correct password! This password would be
``hsrmnssghrd{\textasciigrave}rx'', as can be seen on an ASCII chart. We cannot test during static analysis, but we can
try to understand what should happen and what we would expect to happen. This makes dynamic analysis easier because we
know what to try and which parts to check.}

\subsection{Dynamic Analysis}

\textbf{We will use radare2 as a tool to dynamically analyze the program now. While IDA has some debugging features,
radare2 (commonly referred to as r2) has tons of features to view the variables and stack as you debug a binary, and it
is incredibly helpful in understanding what happens as a binary runs.}

\textbf{One easy way to get to a point in a binary is by manipulating the jumps. Sometimes, it does not matter what you
enter because you can change the register values later on to reach your goals. We will load the binary example into
radare2 and seek to main.}

  
\sitgfx[width=6.5in,height=4.0626in]{reversing-img053.png}{fig:unk}{TODO CAPTION}
 \textbf{Now, recall the IDA graph that we analyzed, and remember that the final jump to either correct or failure.
Before this jump is taken, the ``test eax, eax'' instruction is used to determine if the value in eax is zero or not,
setting the zero flag. Use the arrows to scroll through the main function in radare2 and set a breakpoint at the ``test
eax, eax'' instruction. Now, continue and hit this point.}

  
\sitgfx[width=6.5in,height=4.0626in]{reversing-img054.png}{fig:unk}{TODO CAPTION}
 \textbf{We will change the value in the register eax to be zero. This way, the JNE instruction (jump if not equal to)
will evaluate to false and not be taken. This should print the affirmative response. To do this, type ``:dr eax=0''.
You can see the value of the eax register change to 0. }  
\sitgfx[width=6.5in,height=4.0626in]{reversing-img055.png}{fig:unk}{TODO CAPTION}
 

\textbf{After you do this, step twice to see if the jump is taken or not. You will see that the jump is not taken! By
changing the value in the register you have beaten the challenge! Continue execution to see that you won.}

\textbf{Wait, is it also possible to change the instructions? Yup! This time we will change the JNE instruction into a
NOP, and this will make sure that the jump is not taken and the program flows into the correct section of the program.
To do this, you will load example into radare2 and seek to main. Then, you will scroll through main until the ``jne
0x56624689'' instruction is at the top of the code. Then, you can use ``c'' to go to cursor mode, and tab twice to put
the cursor in the code. Use the arrow keys to put the cursor on the opcodes of the instruction that you want to edit.
Now, type ``a'' and you will be prompted to enter assembler opcodes.  You will write two NOP instructions (0x90) in the
place of the jump not equal instruction.}

  
\sitgfx[width=6.5in,height=4.0626in]{reversing-img056.png}{fig:unk}{TODO CAPTION}
 

  
\sitgfx[width=6.5in,height=4.0626in]{reversing-img057.png}{fig:unk}{TODO CAPTION}
 

\textbf{We use two because a NOP is one byte and a JNE is 2 bytes in length. A NOP instruction stands for ``no
operation'', and the code execution flows through these NOP instructions without changing anything except for the eip.
This is called patching, and it is used to change the flow of programs and redirect execution to wherever you want.
Doing this in radare2 in debug mode will not change this permanently, however, so you can test anything that you want
without having to worry about ruining your binary! As you run the program, you can put any input and still reach the
correct component of the program.}

\textbf{Now, we will put this binary into a different context. Let's say that this is a challenge and the flag is the
password that we input. Now, the input actually matters, so we must find out what the password should be. Open the
binary example in radare2 and seek to main. Here we can see the gets, puts, strlen, and strcmp functions that were seen
in IDA as they are called. They are highlighted into a light green color at the default color setting to stand out as
functions that are called.}

  
\sitgfx[width=6.5in,height=4.0626in]{reversing-img053.png}{fig:unk}{TODO CAPTION}
 

\textbf{In order to see what the puts function is doing, let's put a breakpoint after puts and debug continue until that
point. After doing this, you will see the phrase ``Please enter your password: ``, which shows the string that the puts
call used from the stack as an argument}

\textbf{The string was located at the address [ebx-0x189f]. The square brackets dereference the value inside them in x86
assembly. So if ebx was 0x55555555 then [ebx] would tell what is stored at the address 0x55555555 instead of the
location itself. To display the string at other times during execution before ebx is changed, you can print the string
in the memory at this location using the command ``pxw 0x20 @ ebx-0x189f''. Note that the pxw dereferences the argument
given to them automatically. So to see the value at ebx use ``pxw 0x20 @ ebx'', not ``pxw 0x20 @ [ebx]''. Pxw stands
for print hexadecimal (x) word. A word is 4 bytes in length. }

  
\sitgfx[width=6.5in,height=4.0626in]{reversing-img058.png}{fig:unk}{TODO CAPTION}
 

\textbf{After this, we know that the gets function takes user input because it is a commonly used libc function. To find
out what the purpose of a libc function like gets is, you should use the command ``man gets'' in the command line in a
new terminal window. You should notice that the man page warns the user that gets is an insecure function to use.
Looking back before the gets function is called, we see that the value ebp-0x24 is loaded into eax and pushed to the
stack. This means that the local variable ebp-0x24 is the first argument to gets, and the location of the local
variable where our input is stored. This can be determined by the man page for gets as the first argument, so it is
always a good idea to use the man pages for unknown libc functions. Single step to the call.}

  
\sitgfx[width=6.5in,height=4.0626in]{reversing-img059.png}{fig:unk}{TODO CAPTION}
 

  
\sitgfx[width=6.5in,height=4.0626in]{reversing-img060.png}{fig:unk}{TODO CAPTION}
 

\textbf{Right now, you should use shift s to step over the gets call. This is important because the code for the gets
function is long and complicated. If there is an issue and you land inside of a function like gets and just want to
return to the previous function, use the command dcr to continue until the next ret instruction. Instead of this, you
can use s main to seek back to main, and you can set a breakpoint after gets returns. Then, dc should return you to
this point.}

\textbf{Now let's print 20 bytes, as words, at the location ebp -- 0x24 to check that our input is stored there. After
this, let's try to find the precise location that the input is stored at. To do this, you can use the command ``? ebp
-- 0x24''. The second value shown is the hexadecimal formatted address for this value. As you can tell, this is the
same value as eax, which shows that the gets function returns the value of the pointer to the input.}

  
\sitgfx[width=6.5in,height=4.0626in]{reversing-img061.png}{fig:unk}{TODO CAPTION}
 

\textbf{Note that the same ebp-0x24 value is the argument to sym.imp.strlen. place a breakpoint after the call to strlen
and debug continue. Note that after the call, the value in eax is 0x10 (which is 16, not 10). The this value is place
into the stack variable ebp-0x10. On the left side of the screen, you'll see two arrows. These show that the program
execution will be jumping up and down the instructions here. If you look at the bottom jump command, you notice that
the compare jumps if the value in ebp-0xc is lower then the value in ebp-0x10. So this loop will probably count upwards
by one until it reaches the length of our string.}

  
\sitgfx[width=6.5in,height=4.0626in]{reversing-img062.png}{fig:unk}{TODO CAPTION}
 

\textbf{Looping at the code inside the loop, we see that the loop takes the address of our string and counts upward
through each letter in the string, and adds one to the ascii value in each letter (so `h' becomes `I', `a' becomes `b',
`b' becomes `c').}

  
\sitgfx[width=6.5in,height=4.0626in]{reversing-img063.png}{fig:unk}{TODO CAPTION}
 

\textbf{Next we see that the binary is calling a string compare. Let's see what is being compared to our input.}

  
\sitgfx[width=6.5in,height=4.0626in]{reversing-img064.png}{fig:unk}{TODO CAPTION}
 

\textbf{Well, now we think we may know what the password entered should be. Now, I'll reset the instruction pointer to
the beginning of main by executing dr eip = 0x565fd5dc. (This address may be different for you on your computer.)}

\textbf{To solve the crackme, we will use a python script. We know that the solution should involve every character of
itsnotthiseasy that is decremented by one. The python script is as follows:}

  
\sitgfx[width=6.5in,height=4.0626in]{reversing-img065.png}{fig:unk}{TODO CAPTION}
 

\subsection{Tips and Tricks}

\textbf{ \ \ Now that you have gone through a normal static analysis workflow, you can tell that this could become
incredibly time consuming very quickly. However, there are tons of little tricks that can save you time. Many
constructs are used in similar fashions throughout many C programs, like the way that all for loops are recognizable
for their arrows backwards to previous boxes and a register or variable incrementing. To help you see this, I created a
C program that involved many different data types and control flows. This is similar to a great video I watched by a
Youtuber known as Live Overflow, so I want to make sure to give credit to him and his ideas. I highly recommend that
you watch his binary exploitation series. Because this is not as much about reversing a program without source code, I
provided the source code  for the program. It is in the c\_constructs.c file, which you should have open next to IDA as
you analyze.}

  
\sitgfx[width=3in,height=4.1252in]{reversing-img066.png}{fig:unk}{TODO CAPTION}
 

\textbf{To help you understand what the size is of each character, you can see what kind of memory location they are
stored in. A dword, or double word, is a memory location that is 4 bytes in length. A qword, or quad word, is a memory
location that is 8 bytes in length. As you can tell from the various pointers to memory location, all of which were
created for local variables, all are stored in dwords, qwords, or bytes. The most common of these that you will see are
characters in strings and integers. The character is always represented as a byte, and integers and stored as dwords in
memory. Because of this, if you see bytes being used and referenced frequently in a program, you could infer that
strings and characters are probably being used. If there are dwords being used in a password checking program, it is
more likely that it is some sort of passcode with numbers.}

  
\sitgfx[width=6.4898in,height=3.802in]{reversing-img067.png}{fig:unk}{TODO CAPTION}
 

\textbf{This is an example of a for loop. The variable labelled ``var\_40'' is the counter that loops through the for
loop. When you look quickly at a loop like this, you can tell that this is a loop because of the blue arrow that points
upwards to the box at loc\_563. Also, you can tell that it's a for loop that increments from 0 to 9 because the initial
instruction is ``mov [ebp+var\_40], 0''. Then, it is compared against 9 after being moved into edx, and it is
incremented in the box on the bottom left of the picture above. All of these can be seen by reading through the
assembly language line by line, but looking at the boxes before and after the arrow that goes upwards can reveal a ton
about the functionality.}

  
\sitgfx[width=6.5in,height=3.9374in]{reversing-img068.png}{fig:unk}{TODO CAPTION}
 

\textbf{Here are two other loop examples. The loop on the right is taken first because var\_40 is still 10 from the
previous for loop. Here a while loop is used, which can somewhat be seen by the way that 2 was added to edx instead of
1, but for loops can use increments of 2 as well. You can definitely tell that it is a loop because of the arrow
pointing upwards, and it isn't quite the same as the for loop from before. Then, the box on the left is a do while
loop. This can be seen because of the arrow pointing upwards into the same box. Also, the instructions that start with 
{}``f'' indicate floating point instructions, and these are used when floats or doubles are used in computations.}

  
\sitgfx[width=6.4898in,height=0.6252in]{reversing-img069.png}{fig:unk}{TODO CAPTION}
 

\textbf{A switch can be easily distinguished by the long row of boxes. All of these boxes are aligned and are referenced
from the same box above them. Also, I can identify switches often based on the fact that they are reached by a ``jmp
eax'' instruction rather than a jump based on the comparison of values. Many different values can be stored in eax, and
these pointers can be used based on the case of the switch.}

  
\sitgfx[width=2.8646in,height=1.2083in]{reversing-img070.png}{fig:unk}{TODO CAPTION}
 

\textbf{When a function is called in this program, the add function, the arguments are pushed onto the stack in reverse
order. This is a normal calling convention for C, and it is called the cdecl convention. This stands for ``C
declaration'', and it is used in most C programs that are compiled on a Unix system as ELF files. All of the programs
that we have reversed so far have used the cdecl convention, like the main methods in this program and in the previous
example program. What happens in these is that the arguments are pushed in reverse order after space is created for
these, and then they are located in higher memory than the stack pointer esp once the function is called and the new
stack frame is created. }

  
\sitgfx[width=3.3646in,height=3.7602in]{reversing-img071.png}{fig:unk}{TODO CAPTION}
 

\textbf{In the called function, the new frame is created without subtracting esp at all. This is because no new
variables were created in this function, and the return value was put into eax. This is a standard convention in
assembly language, so you can safely assume for most conventions that eax holds the return value for the previously
called function until anything else is placed in the register.}

  
\sitgfx[width=6.5in,height=4.5311in]{reversing-img072.png}{fig:unk}{TODO CAPTION}
 

\textbf{Here you can see how the hardcoded array is placed into memory. All of the integers are placed into memory
locations located next to each other, within 4 bytes of each other. This helps in traversing the array because the
elements are all 4 bytes away from each other. An important line here is ``mov [ebp+eax*4+var\_58], edx''. This helps
locate each element in the array because the first element of the array is located at ebp+var\_58. Then, the eax times
4 helps traverse from one element to the next based on the value of eax, starting with eax equal to 0. After this, a
string is hardcoded into the array based on the ascii values in reverse order, known as little endian order. Also, as
we mentioned the return values for functions, a value of 0 is moved into eax to indicate that the program was able to
exit without any error!}

\textbf{The best way to learn reverse engineering, especially static analysis, is by creating your own programs with
strange and unique constructs and seeing how the graph in IDA reflects your functions. Knowledge of loops and
conditions makes this much easier, but there are many ways in which different algorithms and functions could be
represented. To practice your reverse engineering skills and learn how program control flows in various ways, I have
provided a famous bomb lab from Carnegie Mellon University. This would be tested by running it or using dynamic
analysis, but you can use static analysis to read each phase and understand how they work. The most important thing is
to be patient, keep an x86 manual nearby, and make sure to look at the bigger picture!}

\section{Exploits}

\textbf{ \ \ We should now look at the functions that are used to see where there could be vulnerabilities. The most
obvious sore thumb is the gets method. This method is not safe to use because it does not hold bounds. You could enter
any amount of characters and overwrite other parts of memory in the program. This holds many options, as you will see
during the dynamic analysis method, and you will also learn about how to reach that secret ``win'' function that we
found in IDA.}

\subsection{Buffer Overflows}

\textbf{This tutorial should give a basic understanding of buffer overflows, shellcode, and what first steps to take
when approaching a ctf-style pwn challenge. The tutorial is for a 64 bit linux (Kali specificially) using radare2
version 2.3.0 and is aimed to be for beginners. However, some experience with assembly and C or C++ programming
languages will help dearly. For anyone reading this without any experience in assembly and C, I recommend reading up on
those topics first.}

\textbf{BuffO\_1.out}

\textbf{Accompanying this tutorial should be a link to several program files. To start off, we will be working with
BuffO\_1.out. A good first step when approaching a binary problem is to scout a little about your file. Run ``file
BuffO\_1.out''}

  
\sitgfx[width=5.8335in,height=0.7665in]{reversing-img073.png}{fig:unk}{TODO CAPTION}
 

\textit{Output of file BuffO\_1.out}

\textbf{ELF stands for Executable and Linkable Format. This really just means the program is built for linux. If this
said PE32+ executable, x86-64 then the file would be a binary built for windows.}

  
\sitgfx[width=5.8335in,height=0.9602in]{reversing-img074.png}{fig:unk}{TODO CAPTION}
 

  
\sitgfx[width=5.8335in,height=3.6457in]{reversing-img075.png}{fig:unk}{TODO CAPTION}
 

\textbf{Now type ``V'' then enter to open visual mode}

  
\sitgfx[width=5.8335in,height=3.6457in]{reversing-img076.png}{fig:unk}{TODO CAPTION}
 

\textit{output from v}

\textbf{The first visual mode is a basic hexdump of the binary. This can be useful at times but we will go for a
different view for now. type ``p'' twice without hitting enter.}

  
\sitgfx[width=5.8335in,height=3.6457in]{reversing-img077.png}{fig:unk}{TODO CAPTION}
 

\textbf{This may be a little bit to take in if this is completely new to you so take a deep breath, but don't worry.
It's not so bad. Now, you can see a view of the instructions in memory that are going to be executed at the bottom of
the display. The rows of 0x00000000's with labels are the registers of the computer and at the top of the screen is a
view of the stack.}

  
\sitgfx[width=5.8335in,height=3.6457in]{reversing-img027.png}{fig:unk}{TODO CAPTION}
 

\textbf{For those new to assembly, registers are like variables for assembly. You can move data into and out of
registers (mov rax, 6)and use the to push and pop data from the stack (pop rax). Several registers have special
purposes. rip is used to point to the next assembly intsruction the computer is going to execute in memory. Right now,
rip is 0x7f3016a9bea0 and if we look at the code at the bottom panel we see this agrees with the topmost instruction in
memory.}

  
\sitgfx[width=5.8335in,height=3.6457in]{reversing-img078.png}{fig:unk}{TODO CAPTION}
 

\textbf{The register rsp is used to point to the top of the stack. As you can see, the address in rsp is 0x7ffc2d515cf0
which agrees with the highest stack address in the top view.}

  
\sitgfx[width=5.8335in,height=3.6457in]{reversing-img079.png}{fig:unk}{TODO CAPTION}
 

\textbf{To see where execution of the program actually begins type the colon : without hitting enter. Sometimes r2 is
glitchy and you may have to hit it twice. You should see a tab for command execution pop up at the bottom of the
screen. Type ``db main'' and hit enter, then ``dc'' and hit enter again.}

  
\sitgfx[width=5.8335in,height=3.6457in]{reversing-img080.png}{fig:unk}{TODO CAPTION}
 

\textbf{main is the execution starting point of programs written in C and C++ for command line. the ``db'' command sets
a breakpoint inside the code of the program. So by typing ``db main'' you set a breakpoint at the beginning of the main
function, which is the start of the program. (Radare2 actually starts a program and breaks before the main function or
the program starts.) Other functions can be used the same way. For example if a program had a function in it called
sayHello() then ``db sayHello'' would put a breakpoint at the beginning of that function. You can enter offsets into
the ``db'' command as well. type ``db main+1'' to set a breakpoint at the second instruction of the main function. The
``dc'' command stands for debug continue, and runs the program until another breakpoint is hit. Breakpoints can also be
set by using the up and down keys to look through memory and then hitting shift + b to place a breakpoint at the
highest instruction in the display. (You cannot use this when the tab for commmand execution is open at the bottom of
the screen. hitting enter without typing any command will get rid of the tab.) Hitting shift + b again will remove the
breakpoint. or using a minus (-) in the ``db'' command. For example ``db -main'' will remove the breakpoint at the
start of main.}

  
\sitgfx[width=5.8335in,height=3.6457in]{reversing-img081.png}{fig:unk}{TODO CAPTION}
 

\textbf{Set breakpoints after each of the ``call'' instructions in main and you should see b's next to their
instructions. If you run ``dc'' now, you will see the execution of the program. First, ``Password please:'' is printed
to the screen}

  
\sitgfx[width=5.8335in,height=3.6457in]{reversing-img082.png}{fig:unk}{TODO CAPTION}
 

\textit{output from the program that says Password Please}

\textbf{running ``dc'' again opens a prompt for user input}

\textbf{ans the program is built for linux. If this said PE32+ executable, x86-64 then the file would be a binary built
for windows.}

  
\sitgfx[width=5.8335in,height=0.9602in]{reversing-img074.png}{fig:unk}{TODO CAPTION}
 

\textbf{Now run ``checksec Buff0\_1.out''}  
\sitgfx[width=5.8335in,height=3.6457in]{reversing-img083.png}{fig:unk}{TODO CAPTION}
 

\textit{Output of checksec BuffO\_1.out shows different mitigation techniques for exploitation.}

\textbf{Here we see a list of common mitigation techniques to make programs harder to hack. The three labeled RWX, NX,
and Stack will be focused on today but I recommend reading up on RELRO, PIE (Postition Independent Executable) and how
the Global Offset Table works for further development after you have a grasp of the others. To fully understand how
each of these mitigation techniques work and how they can be circumvented, we need a little background.}

\subsection{Mitigation Techniques}

\textbf{In checksec's output, we saw the line marked Stack: No canary found. Canary's are special secret numbers that
the computer places onto the stack. This way, if someone overwrites a section of memory they are not supposed to, the
computer can check to see that the canary's are still intact, and if they have been changed, abort the program. Having
no canaries in our program is nice and we will not have to deal with them.}

\textbf{The stack is one of several regions of memory. Each region of memory can be marked as readable (r), writable
(w), or executable (x). Memory can only be accessed if it is marked appropriately. As a attacker, seeing memory marked
RWX is a good sign because we can know we can write vulnerable code to the region and then execute it. Later in this
tutorial, we will overwrite the stack with ``shellcode'' of our own and execute it.}

\textbf{The output from checksec underneath ``Stack: No canary found'' says ``NX: NX disabled''. You will find that the
stack is often a big target for executing shellcode on. Because of this, checksec specially tells you if the stack is
labeled executable or not. NX stands for No-Execute. Having NX disabled (talk about a double negative) means we can run
wild with shellcode on the stack!}

\textbf{There is another mitigation technique that is not listed in checksec's output but is worth mentioning. Address
Space Layout Randomization (ASLR) rearranges the locations of areas of memory in computer programs inside a computer.
This may not mean much now but keep it in mind for later. You'll find when you a reversing programs that addresses
change each time you close and reopen the program. If you want to turn off ASLR you can execute:}

\textbf{echo 0 {\textgreater} /proc/sys/kernel/randomize\_va\_space}

\textbf{Don't develop a crutch by relying on this all the time though! Also, I would not recommend permanently disabling
it on your computer for security reasons.}

\subsection{Running Radare2}

\textbf{Now for the good stuff!}

\textbf{To open up the binary with radare2 run ``r2 -d BuffO\_1.out''. The -d means open the file using debugging mode.
You should see a output similar to below:}

  
\sitgfx[width=5.8335in,height=3.6457in]{reversing-img084.png}{fig:unk}{TODO CAPTION}
 

\textbf{Radare2 can perform some automatic analysis of the binary that is helpful. Type ``aaa'' and hit enter to analyze
the binary. Sometimes, creators of binaries can mangle this and make analyzing more difficult. Or if the binary is
extremely big, this may take a few seconds. However, it is normally worth a shot.}

  
\sitgfx[width=5.8335in,height=3.6457in]{reversing-img075.png}{fig:unk}{TODO CAPTION}
 

\textbf{Now type ``V'' then enter to open visual mode}

  
\sitgfx[width=5.8335in,height=3.6457in]{reversing-img076.png}{fig:unk}{TODO CAPTION}
 

\textit{output from v}

\textbf{The first visual mode is a basic hexdump of the binary. This can be useful at times but we will go for a
different view for now. type ``p'' twice without hitting enter.}

  
\sitgfx[width=5.8335in,height=3.6457in]{reversing-img077.png}{fig:unk}{TODO CAPTION}
 

\textbf{This may be a little bit to take in if this is completely new to you so take a deep breath, but don't worry.
It's not so bad. Now, you can see a view of the instructions in memory that are going to be executed at the bottom of
the display. The rows of 0x00000000's with labels are the registers of the computer and at the top of the screen is a
view of the stack.}

  
\sitgfx[width=5.8335in,height=3.6457in]{reversing-img027.png}{fig:unk}{TODO CAPTION}
 

\textbf{For those new to assembly, registers are like variables for assembly. You can move data into and out of
registers (mov rax, 6)and use the to push and pop data from the stack (pop rax). Several registers have special
purposes. rip is used to point to the next assembly intsruction the computer is going to execute in memory. Right now,
rip is 0x7f3016a9bea0 and if we look at the code at the bottom panel we see this agrees with the topmost instruction in
memory.}

  
\sitgfx[width=5.8335in,height=3.6457in]{reversing-img078.png}{fig:unk}{TODO CAPTION}
 

\textbf{The register rsp is used to point to the top of the stack. As you can see, the address in rsp is 0x7ffc2d515cf0
which agrees with the highest stack address in the top view.}

  
\sitgfx[width=5.8335in,height=3.6457in]{reversing-img079.png}{fig:unk}{TODO CAPTION}
 

\textbf{To see where execution of the program actually begins type the colon : without hitting enter. Sometimes r2 is
glitchy and you may have to hit it twice. You should see a tab for command execution pop up at the bottom of the
screen. Type ``db main'' and hit enter, then ``dc'' and hit enter again.}

  
\sitgfx[width=5.8335in,height=3.6457in]{reversing-img080.png}{fig:unk}{TODO CAPTION}
 

\textbf{main is the execution starting point of programs written in C and C++ for command line. the ``db'' command sets
a breakpoint inside the code of the program. So by typing ``db main'' you set a breakpoint at the beginning of the main
function, which is the start of the program. (Radare2 actually starts a program and breaks before the main function or
the program starts.) Other functions can be used the same way. For example if a program had a function in it called
sayHello() then ``db sayHello'' would put a breakpoint at the beginning of that function. You can enter offsets into
the ``db'' command as well. type ``db main+1'' to set a breakpoint at the second instruction of the main function. The
``dc'' command stands for debug continue, and runs the program until another breakpoint is hit. Breakpoints can also be
set by using the up and down keys to look through memory and then hitting shift + b to place a breakpoint at the
highest instruction in the display. (You cannot use this when the tab for command execution is open at the bottom of
the screen. hitting enter without typing any command will get rid of the tab.) Hitting shift + b again will remove the
breakpoint. or using a minus (-) in the ``db'' command. For example ``db -main'' will remove the breakpoint at the
start of main.}

  
\sitgfx[width=5.8335in,height=3.6457in]{reversing-img081.png}{fig:unk}{TODO CAPTION}
 

\textbf{Set breakpoints after each of the ``call'' instructions in main and you should see b's next to their
instructions. If you run ``dc'' now, you will see the execution of the program. First, ``Password please:'' is printed
to the screen}

  
\sitgfx[width=5.8335in,height=3.6457in]{reversing-img082.png}{fig:unk}{TODO CAPTION}
 

\textit{output from the program that says Password Please}

\textbf{running ``dc'' again opens a prompt for user input}

  
\sitgfx[width=5.8335in,height=3.6457in]{reversing-img085.png}{fig:unk}{TODO CAPTION}
 

\textbf{running ``dc'' again will exit the program. At this point you can enter ``ood'' to restart the program. However,
you will have to re-enter your breakpoints.}

\textbf{You can also step through a program by hitting shift and s. This will execute one instruction at a time and show
how the program changes with each step. Holding shift will make you jump over a calls to other parts of the program. If
you let go of shift and step at the instruction ``call sym.imp.puts'' for example, you will see the assembly for the
sym.imp.puts function. To get back so main, open the command tab and enter ``s main'' which stands for seek main. This
will move you back to the main function where you can set a breakpoint after the ``call sym.imp.puts'' instruction and
then run ``dc'' to finish the call to sym.imp.puts. At this point, you should hit q and then q again to exit radare2
and make sure you can run these commands again. Feel free to play around with entering a command then ``?'' to try to
get help.}

\subsection{Explanation of the assembly}

\textbf{The function prologue}

\textbf{Now, I will explain the assembly instruction of the program to give a understand of why some of the code looks
like it does. This tutorial is meant to get complete beginners started, Unfortunately, I cannot give a full tutorial on
programming in x86 assembly as this would take a book to explain. However, this should get you up and running with some
basic assembly and hopefully will allow you to smoothly transition to other tutorials.}

\textbf{The first instruction is ``push rbp''.}

  
\sitgfx[width=5.8335in,height=0.7035in]{reversing-img086.png}{fig:unk}{TODO CAPTION}
 

\textbf{The instruction pushes whatever data is in the register rbp onto the stop of the stack. Take a mental note of
the value held in the rbp register then hit ``s'' to step over this instruction and you will see that value at the top
of the stack.}

  
\sitgfx[width=5.8335in,height=3.6457in]{reversing-img087.png}{fig:unk}{TODO CAPTION}
 

\textbf{Next is the ``mov'' instruction. mov is short for move and takes two operands. Whatever is in the second operand
is moved into the first. So for example to move 6 into rax you would see the instruction}

\textbf{{}``mov rax, 0x6''}

\textbf{The ``0x'' before the 6 means the number is in hexadecimal. Instead of counting 0, 1, 2, {\dots} 8, 9, 10 the
computer uses 16 symbols 0x0, 0x1, 0x2, {\dots} 0x8, 0x9, 0xa, 0xb, 0xc, 0xd, 0xe, 0xf, 0x10. Where 0xa corresponds to
10. 0xb corresponds to 11, etc.}

\textbf{The instruction ``mov rbp, rsp'' takes whatever value is in rsp (the top of the stack) and places it into rbp.}

  
\sitgfx[width=5.8335in,height=3.6457in]{reversing-img088.png}{fig:unk}{TODO CAPTION}
 

\textit{shows the execution after mov rbp, rsp}

\textbf{Finally hit ``s'' to run ``sub rsp, 0x70''. This command subtracts 0x70 (which is 7 * 16 = 112) bytes from the
value in rsp. The three commands}

\textbf{{}``push rbp''\newline
{}``mov rbp, rsp''\newline
{}``sub rsp, 0x70''}

\textbf{together are used to set up functions in assembly. They are called the function prologue. Pushing rbp and then
moving the value of rsp into rbp before you subtract from the stack allows you to keep track of where the stack frame
previously was before it grows. By subtracting 0x70 from the rsp register (which points to the top of the stack) you
increase the size of the stack by 112 bytes. Executing these three instructions right after a call instruction (like
the call sym.imp.puts instruction inside main) you efficiently set up new stack frames for each function. Then once a
function is finished the stack frame is returned to it's previous point by reversing these three commands. Because this
concept is so important I will illustrate it further with a basic example.}

\textbf{Lets say the stack frame currently is 0x30 bytes}

\textbf{0x00100: rsp is here\newline
0x00110:\newline
0x00120:\newline
0x00130: rbp is here}

\textbf{and a new function is called in the program:}

\textbf{0x0001: call new\_function\newline
0x0002: do stuff after the new\_function{\dots}\newline
{\dots} 0x0005: new\_function starts here}

\textbf{The call instruction changes the execution of a program. Whenever a function is called in C or C++ this compiles
down to a call instruction or similar in the assembly. The call instruction pushes the next instruction after the call
onto the stack. So in our example above, 0x0002 will be pushed onto the stack. Then the programs instruction pointer
(rip) jumps to the beginning of the new\_function. So the stack would now look like}

\textbf{0x00090: rsp (top of stack)\newline
0x00100: 0x0002\newline
0x00110:\newline
0x00120:\newline
0x00130: rbp is here}

\textbf{and the instruction pointer would be at the beginning of new\_function}

\textbf{new\_function()\newline
0x0005: rip is here\newline
0x0006: do some stuff{\dots}}

\textbf{After a call instruction is executed normally the three instructions ``mov rbp, rsp'' ``pop rbp'' and ``sub rsp,
some\_amount'' will be executed. So lets see the three commands be executed at the beginning of new\_function.}

\textbf{new\_function()}

\textbf{0x0005: push rbp\newline
0x0006: mov rbp, rsp\newline
0x0007: sub rsp, 0x30 ;(grow the size of the stack by 0x30)}

\textbf{First, push rbp is called and the stack will look like this:}

\textbf{0x00080: rsp is here\newline
0x00090: 0x00130 (rbp's value)\newline
0x00100: 0x0002 (the instruction to be executed after new\_function returns)\newline
0x00110:\newline
0x00120:\newline
0x00130: rbp is here}

\textbf{Then ``mov rbp, rsp'' is called and the value in rsp is copied over to the rbp register}

\textbf{0x00080: rsp and rbp are here\newline
0x00090: 0x00130 (rbp's old value of 0x00130)\newline
0x00100: 0x0002 (the instruction to be executed after new\_function returns)\newline
0x00110:\newline
0x00120:\newline
0x00130:}

\textbf{finally 0x30 is subtracted from rsp and the stack grows.}

\textbf{0x00050: rsp is here\newline
0x00060:\newline
0x00070:\newline
0x00080: rbp is here\newline
0x00090: 0x00130 (rbp's previous value)\newline
0x00100: 0x0002 (the instruction to be executed after new\_function returns)\newline
0x00110:\newline
0x00120:\newline
0x00130:}

\textbf{If you take a look at the bottom of the main function in radare2 you will see two instructions. leave and ret}

  
\sitgfx[width=5.8335in,height=3.6457in]{reversing-img089.png}{fig:unk}{TODO CAPTION}
 

\textbf{These instructions are the ones that reverse the first three and return the stack frame to it's previous
position. The leave instruction is basically equivalent to running}

\textbf{mov rsp, rbp\newline
pop rbp}

\textbf{which would first move rbp's value into rsp}

\textbf{0x00060:\newline
0x00070:\newline
0x00080: rbp and rsp are now here\newline
0x00090: 0x00130\newline
0x00100: 0x0002\newline
0x00110:\newline
0x00120:\newline
0x00130:}

\textbf{then pop the value at the top of the stack into rbp}

\textbf{0x00090: rsp is now here\newline
0x00100: 0x0002 (the next instruction after new\_function returns)\newline
0x00110:\newline
0x00120:\newline
0x00130: rbp is now back to it's original place}

\textbf{finally, ret is called. ret takes whatever value is at the top of the stack and movs it into rip, the
instruction pointer. Now the stack is back to it's original frame and the instruction pointer is directly after the
call new\_function instruction}

\textbf{0x00100: rsp is here\newline
0x00110:\newline
0x00120:\newline
0x00130: rbp is here\newline
{\dots}\newline
0x0001: call new\_function\newline
(rip is here)\newline
0x0002: do stuff after new function}

\textbf{I would bet that you are now thinking, great! Why is this important? Hacking often revolves around abusing a
existing structure in a computer to accomplish a unexpected result. Imagine what would happen if the memory on the
stack was corrupted? For example, lets say I stuff tons of ``A'' values into the stack which is equivalent to the
hexadecimal number 0x41.}

\textbf{0x00050: rsp is here\newline
0x00060: AAAAA\newline
0x00070: AAAAA\newline
0x00080: AAAAA (rbp used to be here)\newline
0x00090: AAAAA (0x00130 used to be here which was rbp's previous value)\newline
0x00100: AAAAA (0x0002 was here which was the instruction to be executed after new\_function returns)\newline
0x00110: AAAAA\newline
0x00120:\newline
0x00130:}

\textbf{Now when the computer calls leave and return, it tries to load the value AAAA into the rbp and rip registers
which crash the program. (muahaha, breaking computers is so much fun). What would happen if instead of AAAAA, we spread
a memory address of a function we wanted to call over the stack? Instead of crashing, the program instruction pointer
jumps to this unexpected function and starts running it. This is called a buffer overflow vulnerability and can be used
to call functions that spawn shells, connect to the internet, or other nefarious deeds.}

\subsection{Exploitation}

\textbf{Lets try to buffer overflow our program Buff0\_1.out!}

\textbf{In main, the program runs}

\textbf{{}``lea rdi, qword [rip + 0x118]''\newline
call sym.imp.puts}

\textbf{qword stands for quadword and is a certain size memory chunk in the computer. lea stands for load effective
address. So, this command is loading the address of the quadword at rip+0x118 into the register rdi. You can check by
stepping this instruction and seeing the value of (rip + 280) be moved into rdi. The register rdi is used as the first
argument to a function when calling a function in assembly. The order of arguments passed to a function in a 64bit call
are rdi, rsi, rdx, rcx, r8, r9, then any more arguments are placed on the stack. (Note that this is different from a
32bit computer, which will described later.) You can use the command ``pxq 0x20 @ rdi'' to print 0x20 bytes of hex
(hence the x in pxq) at the value held in rdi. If you do this, you will see ``Password please:'' stored in memory as a
string at that location. By calling sym.imp.puts after this instruction the compiler passes the value ``Password
please'' to the puts function, which prints it to the screen.}

  
\sitgfx[width=5.8335in,height=3.6457in]{reversing-img090.png}{fig:unk}{TODO CAPTION}
 

\textbf{After the program prints ``Password please'' it goes to accept input from the user. ``lea rax, qword [rbp -
0x70]'' loads the stack address 0x70 bytes above rbp into rax. This is how function variables are stored in memory.
Inside of a function when a variable is declared, like this}

\textbf{int function() \{ int local\_var; local\_var = 5; \}}

\textbf{local\_var is a address above the rbp like rbp - 0x70. The second instruction would compile to something similar
to ``mov rax, 0x5'' then ``mov [rbp-0x70], rax''}

\textbf{the value of [rbp-0x70] is then moved from rax into rdi. So our local variable is the first argument to the gets
function. Next, a 0 is moved into eax (which clears rax) and sym.imp.gets is called. If place a breakpoint after ``call
sym.imp.gets'' and dc you will see ``Password please'' printed to screen.}

  
\sitgfx[width=5.8335in,height=2.9654in]{reversing-img091.png}{fig:unk}{TODO CAPTION}
 

\textbf{If we open up a new terminal window and type ``man gets'' we can see the linux manual page for the gets
function. Gets takes input from the stdin until a newline character or EOF is reached. Lets try throwing a bunch of A's
into the local variable with a few B's and C's at the end and see if we can overwrite any important stack addresses.
Since I am looking to overwrite the return address of main, I'll start with 0x70 bytes (or 7 * 16 = 112) of A's then
throw 16 B's and 16 C's. Place a breakpoint after the second call instruction then hit dc and you will see you can
enter user input in the command tab.}

  
\sitgfx[width=5.8335in,height=3.6457in]{reversing-img092.png}{fig:unk}{TODO CAPTION}
 

\textit{Show my input of many As in the command tab}

\textbf{and just like that, our stack is full of 0x41s. If you would like to examine the stack, you can hit ``c'' (not
in the command window) and then use up and down to move through the stack. Hitting ``c'' again will bring you back to
the instructions window.}

  
\sitgfx[width=5.8335in,height=3.6457in]{reversing-img093.png}{fig:unk}{TODO CAPTION}
 

\textbf{Now if we hit dc again, we see our program crashes and returns signal 11. Lets rerun the binary and find the
return address for main, and what As exactly overwrite that address. Normally, you could see when a function is called
to find the next instruction pointer you can just look at the address of the instruction underneath the call
instruction. Because I am unsure where main is called from, I can simply put a break mark on the return instruction,
continue execution until then, then step and examine the address we return to after main. Once I know the address I
return to, I look at the stack for where that address is relative to rbp.}

\textbf{Here is at the end of main.}\newline
  
\sitgfx[width=5.8335in,height=3.6457in]{reversing-img094.png}{fig:unk}{TODO CAPTION}
 

\textbf{After stepping, we see the next instruction in memory.\newline
}  
\sitgfx[width=5.8335in,height=3.6457in]{reversing-img095.png}{fig:unk}{TODO CAPTION}
 

\textbf{Now, we know the exact location of our return address and can calculate it. If you want to do a quick
calculation in radare, you can hit ``:'' to go into command mode and then type a ``?'' followed by some mathematical
expression. Here I enter 0x70 to see it's decimal value but also you can enter in addition, multiplication, and
subtraction.}

  
\sitgfx[width=5.8335in,height=3.6457in]{reversing-img089.png}{fig:unk}{TODO CAPTION}
 

\textbf{After a little testing I find that inputting 0x78 A's followed by 6 B's exactly overwrites my return address.
So, now if instead of overwriting my return address with 6 B's, I could overwrite append the address of a function
after the 0x78 A's. If you scroll to the code above main's function in the binary by hitting the up key a few times,
you will notice a function labeled sym.never\_called(). I wonder who put that function there :).}

  
\sitgfx[width=5.8335in,height=3.6457in]{reversing-img096.png}{fig:unk}{TODO CAPTION}
 

\textbf{Currently, the address of never called is 0x564ea17e26ca. (Note that this address will be different for you and
will change each time you restart the program. You will have to check it's value each time your restart the program.)
If you try to directly paste this as a string into the radare input after ``Password please:'' you will find the
program does not jump to the new function. The reason is the program is accepting input as a string. We need to enter
the bytes directly into the input. There are two option to do this. The first option is to create a file with our input
by opening up a new terminal window (cntrl shift t if you are using terminator) and entering}

\begin{verbatim}
echo -en "AAAAAAAAAAAAAAAAAAAAAAAAAAAAAAAAAAAAAAAAAAAAAAAAAAAAAAAAAAAAAAAAAAAAAAAA
AAAAAAAAAAAAAAAAAAAAAAAAAAAAAAAAAAAAAAAA564e17e26" > foo.txt
\end{verbatim}

\textbf{the -en removes the newline from the echo command output and treats the input as bytes.}

\textbf{Then create another file called foo.rr2 and enter into it:}

\textbf{stdio=/dev/pts/0\newline
stdin=./foo.txt}

\textbf{Double check by entering ``tty'' into the terminal windows before opening radare2 that the output is /dev/pts/0.
If it is a different number besides 0 use that number. (You can see in the screenshot I changed mine to /dev/pts/1)}

\textbf{finally run radare2 with the command ``r2 -d filename -e dbg.profile=foo.rr2''}

\textbf{Now, whatever bytes are stored in foo.txt will be entered as the input automatically in radare2.}

  
\sitgfx[width=5.8335in,height=3.6457in]{reversing-img097.png}{fig:unk}{TODO CAPTION}
 \textbf{ }

\textbf{The second option for entering input into radare2 is by pwntools from python. pwntools is a library full of
tools useful for binary reversing. This makes it extremely useful and therefore the more useful solution to our
predicament.\newline
First open up two side by side terminal windows. In one type ipython to open up ipythons command interface. Then enter
``from pwn import *{\textquotedbl} and ``p = process(`./BuffO\_1.out')''}

  
\sitgfx[width=5.8335in,height=3.6457in]{reversing-img098.png}{fig:unk}{TODO CAPTION}
 

\textbf{now before you do anything more in python. open up radare with ``r2 -d pid'' where pid is the process id of the
program you just ran in pwntools. (it will say the pid when it starts)}

  
\sitgfx[width=5.8335in,height=3.6457in]{reversing-img099.png}{fig:unk}{TODO CAPTION}
 

\textbf{In the command tab seek to main using ``s main'' it is import that you do not type ``dc'' right now because
radare2 does not realize the process was already started by ipython. It does not know where the instruction pointer is
located.}

  
\sitgfx[width=5.8335in,height=3.6457in]{reversing-img100.png}{fig:unk}{TODO CAPTION}
 

\textit{entering s main after starting radare2}

\textbf{It is actually waiting inside the gets function because user input has not been provided to the program yet. In
order to use this technique you have to put a breakpoint after where user input is accepted in a binary.}

  
\sitgfx[width=5.8335in,height=3.6457in]{reversing-img101.png}{fig:unk}{TODO CAPTION}
 

\textit{show breakpoint after gets in main}

\textbf{There! Now when we type ``dc'' in radare2, it ``catches up'' with the actual process run by ipython and also
waits for user input}

  
\sitgfx[width=5.8335in,height=3.6457in]{reversing-img102.png}{fig:unk}{TODO CAPTION}
 

\textit{show radare2 caught up with pwntools}

\textbf{Right now, radare2 is hanging because the process will not continue until we send it input through pwntools. To
send data to the binary in ipython, we use p.send() or p.sendline(). Sendline will automatically add a newline
character ``'' to the input we give it, which we need for this program since we had to hit enter for gets() to accept
our input.}\newline
  
\sitgfx[width=5.8335in,height=3.6457in]{reversing-img103.png}{fig:unk}{TODO CAPTION}
 

\textbf{Now when we hit enter to execute p.sendline() we will see radare2 hit a breakpoint after the gets function, and
our stack is full of A's. Notice how in python we could just type ``A''* 0x70 instead of copy pasting tons of letters
around.}

  
\sitgfx[width=5.8335in,height=3.6457in]{reversing-img104.png}{fig:unk}{TODO CAPTION}
 

\textit{Show how execution of sendline in pwntools filled stack with As}

\textbf{Only there is a problem here. We have made a small mistake. Programs store data in what is referred to as
little-endian form. If we look at the address I just entered by string to our program ``55813b22864'', on the stack it
is reversed! little-endian form just reverses the order of bytes on the stack. A quick fix to this is to postfix [::-1]
to our string input in ipython. Try restarting the above commands for practice using pwntools to return to this state
but now add the postfix [::-1].}

  
\sitgfx[width=5.8335in,height=3.6457in]{reversing-img105.png}{fig:unk}{TODO CAPTION}
 

\textbf{Now lets set another breakpoint at the ret instruction. Notice that the address is loaded in memory correct
now.}

  
\sitgfx[width=5.8335in,height=1.922in]{reversing-img106.png}{fig:unk}{TODO CAPTION}
 

\textit{Show correct memory address}

\textbf{And we step again.}

  
\sitgfx[width=5.8335in,height=3.6457in]{reversing-img107.png}{fig:unk}{TODO CAPTION}
 

\textbf{And if all goes well, we jump to the beginning of our never\_called function.}

  
\sitgfx[width=5.8335in,height=3.6457in]{reversing-img096.png}{fig:unk}{TODO CAPTION}
 

\textit{Show the instruction pointer at the beginning of never called}

\textbf{hit ``dc'' and you'll note that the program finishes with signal 11. Now go into ipython and enter p.read()}

  
\sitgfx[width=5.8335in,height=3.6457in]{reversing-img108.png}{fig:unk}{TODO CAPTION}
 

\textit{Show output from p.read() instructions}

\textbf{p.read() and p.readline() are used in python to read data from a programs standard output. Note that your output
may look a little different, since I called p.read before and after executing the never\_called function.}

{\centering
\textbf{Differences between 64 and 32 bit}
\par}

\textbf{In some parts of the tutorial, it was noted some topics would look a little different between 32 and 64 bit
computers. Here, I will give a little explanation about what differences you will see. }

\textbf{First off, in a 64 bit computer, the registers hold 64 bits, while in a 32 bit computer, they hold 32. }

\textbf{The 32 bit registers start with e's (eax, ebx, ecx, edx, esi, edi etc) rather than r's (rax, rbx, rcx, rdx) They
work generally the same way as 64 bit registers however you cannot store a 64 bit value in a 32 bit register. The
opposite is true however. 32 bit values can be moved into a 64 bit register. Also, on a 64 bit computer, you can access
the first 32 bits of a register by using 32 bit register names. For example, I can execute ``move eax, 5'' on a 64 bit
machine. You can also access the first 16 bits and second 16 bits of eax with al and ah. The high and low 16 bits for
other registers follow a similar naming pattern. }

\textbf{When you are calling a function in 32 x86 assembly, you do not use the registers to put arguments into a
function. Instead, arguments are pushed to the stack in reverse. }

\end{document}
