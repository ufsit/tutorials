\textbf{Reverse Engineering in infoSec, as simply as we can put it, is to take a
}\textbf{\textcolor[rgb]{0.21960784,0.4627451,0.11372549}{binary, }}\textbf{which is just }

\textbf{another word for a computer program, and figure out what it's trying to do. You run binaries all the time in day
to day life - when you run your browser, when you run Microsoft Word, when you check the weather on your phone. Every
time you run a program or application, a binary file made of 1s and 0s tells your device exactly what it needs to do.
\newline
\newline
If you have written code before, you know that before it can do anything, you have to turn that into a program/binary so
that you can use it. Before that your code is just a useless file. But what if you want to know what's happening under
the covers when you use Word, or Chrome, or anything else running on your computer? This is done only by Reverse
Engineering.  \newline
\newline
Let's start up our Kali Linux and write up our own code get a better idea. The Linux terminal allow you to access most
of the applications in your environment from the command line. For windows users, it is equivalent to the command
prompt application (cmd). Access your terminal by searching ``Terminator'' in the Activities section of the toolbar
}or\textbf{ selecting the red icon on the dash that looks like }
\includegraphics[height=0.2689in]{FINALWORKINGDOCFORMERLYPRECURSOR-img001.png}
 \textbf{. Here you should see a window open with your current user and location in the file system.\newline
}
